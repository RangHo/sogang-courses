\documentclass{article}

\usepackage[margin=0.5in]{geometry}
\usepackage{kotex}

\usepackage{multicol}
\setlength{\parskip}{\baselineskip}
\usepackage{graphicx}

\usepackage{tcolorbox}
\tcbuselibrary{listings}

\usepackage{synttree}

\usepackage{textcomp}

\title{언어의 이해 중간고사}
\author{이주헌 (20191629)}
\date{2019년 10월 24일}

\begin{document}
    
    \pagenumbering{arabic}
    \maketitle
    
    \section{문제 1번}
    
        \emph{보기} (ㄱ)에서 주어진 단어 3개를 각각의 뜻을 가진 작은 단위로 나누면 다음과 같다.
        \begin{enumerate}
            \item truck driver \textrightarrow{} truck + driver \textrightarrow{} (명사 + 명사)
            \item 밤낮 \textrightarrow{} 밤 + 낮 \textrightarrow{} (명사 + 명사)
            \item 종이호랑이 \textrightarrow{} 종이 + 호랑이 \textrightarrow{} (명사 + 명사)
        \end{enumerate}
        비슷하게, \emph{보기} (ㄴ)에서 주어진 단어 3개를 각각의 뜻을 가진 작은 단위로 나누면 다음과 같다.
        \begin{enumerate}
            \item 듣보다 \textrightarrow{} 듣 + 보 + 다 \textrightarrow{} (용언 + 용언 + 어미)
            \item 검푸르다 \textrightarrow{} 검 + 푸르 + 다 \textrightarrow{} (용언 + 용언 + 어미)
            \item 오르내리다 \textrightarrow{} 오르 + 내리 + 다 \textrightarrow{} (용언 + 용언 + 어미)
        \end{enumerate}
        여기에서 \emph{-다}는 형식 형태소로, 실질적인 의미를 가지지 않으므로 무시한다.
    
        두 \emph{보기}에서 알 수 있듯이, 인간 언어는 같은 품사의 단어를 서로 붙임으로서 단어의 뜻을 보충하거나 바꾸는 특징이 있다.
    
    \section{문제 2번}
        
        인간의 모든 심리적인 기제는 무수히 많은 신경 세포의 집합인 뇌에 의해 결정된다. 그리고, 생물학적으로 이야기했을 때, 뇌의 구조는 전적으로 유전자에 의해 결정된다. (인간이 태어난 이후 후천적으로 생성되는 뇌의 구조는 포함하지 않는다.) 따라서, 인간이 선천적으로 가지고 태어나는 모든 심리적인 기제는 전적으로 유전자에 의해 결정된다고 할 수 있다.
        
        그런데, 아동의 언어 습득 양상을 관찰해 보면,
        \begin{enumerate}
            \item 아동의 언어적 지식, 특히 문법적인 지식은 학습에 선행하는 것으로 보인다.
            \item 아동은 마치 박쥐가 선천적으로 초음파를 사용해서 길을 찾는 법을 알듯, 본능적으로 주변 환경으로부터 언어를 학습하는 것으로 보인다.
        \end{enumerate}
        
        위 관찰 결과로부터, 인간의 언어 습득은 선천적인 것이라고 볼 수 있다. 위의 명제로부터, 인간이 선천적으로 가지고 태어나는 모든 심리적인 기제는 유전자에 의해 결정되므로, 인간의 선천적 언어 습득 능력 또한 유전자에 의해 결정된다고 가설을 세울 수 있다.
        
    \section{문제 3번}
    
        보편 문법(Universal Grammar)은 언어학자 노엄 촘스키(Noam Chomsky)가 주장한 언어 습득을 위한 뇌의 기제 또는 선천적 지식으로, 어린이들이 언어를 배우는 데 사용된다. 반면, 심리 문법(Mental Grammar)은 개인이 학습한 특정 언어 패턴의 모음, 또는 그 패턴을 인식하는 기제이다.
        
        어린이는 언어를 습득할 때, 보편 문법을 사용해서 주위 환경에서 사용하는 언어의 패턴을 분석, 내재화시켜 자신만의 심리 문법을 구축한다.
        
    \section{문제 4번}
    
        \subsection{문제 4-(1)}
            
            \begin{enumerate}
                \item (ㄱ) [-nasal] \textrightarrow{} [+nasal] / \_+nasal
                \item (ㄴ) [+coronal, -dorsal, -delayed release] \textrightarrow{} [-coronal, +dorsal, +delayed release] / \_-consonental, +high, -round
            \end{enumerate}
            
        \subsection{문제 4-(2)}
            
            비음 동화 현상은 자음과 자음의 연결으로 인해 발생하는 동화 현상인 반면에, 구개음화 현상은 자음이 특정 모음을 만남으로서 발생하는 현상이다. 그러나, 위에 서술한 규칙은 여러 언어의 예시를 기반으로 하지 않았기 때문에 변별자질로 인한 동화 현상이 모든 언어에서 관찰된다고 할 수는 없다. 이 점은 다른 언어의 예시를 더 알아봄으로서 해결할 수 있다. 또한 구개음화 규칙에서는 한국어에서 구별되는 음소인 [ㅈ]과 [ㅊ], [ㅈ]과 [ㅊ]의 정확한 대응 관계를 변별자질만을 활용해서 알기 어렵다. 이 경우는 성대진동 시작 시간 등의 다른 요소도 고려해야 할 것이다.
    
    \section{문제 5번}
    
        언어 학습은 선천적으로, 또 상대적으로 극히 적은 양의 사례로부터 이루어진다는 점에서 일반 학습과 차이가 있다. 기본적으로 인간은 학습을 할 때, 주어진 예시 또는 사례로부터 패턴을 찾아내어 기억하고, 더 나아가 그 패턴을 응용하는 방법을 익힌다. 그러나, 이런 방식의 패턴 학습은 주관적인 것이기 때문에, 같은 사례를 학습하더라도 개개인 간의 해석 차이가 생길 수 있다. 그 예로, 지구에서 볼 수 있는 천체의 운동을 해석하는 방식이 천동설과 지동설 두 가지로 나뉜 점을 들 수 있다. 각 가설의 정확성을 떠나서, "천체의 움직임"이라는 같은 현상을 두 가지 방식으로 설명했다는 점은 눈여겨볼 만하다.
        
        그러나 언어 학습은 일반 학습에 필요한 것보다 극히 적은 양의 사례(데이터)로부터 패턴을 추출해 낼 수 있다는 점이 특징이다. 거기다 세계의 여러 언어를 비교해 봤을 때, 가능한 문법의 개수가 지극히 제한적이다. 만약 언어 학습이 일반적 학습과 같은 기제로 이루어진다면 문법 패턴이 훨씬 더 다양하게 나타날 것이다. 이것이 바로 "언어 습득의 패러독스(The Paradox of Language Acquisition)"이다. 이렇게 선천적으로, 또 효과적으로 이루어지는 언어 학습을 설명하기 위해서는 인간에게는 언어 학습을 위한 선천적 기제를 상정해야만 한다.
        
        이러한 점을 볼 때, 언어 학습은 일반 학습과 다른 기제를 사용한다는 점은 명확해 보인다. 따라서, "언어 습득의 패러독스"를 해결하기 위한 가장 좋은 방법은 인간의 뇌에는 어떠한 선천적 언어 습득 장치가 있음을 상정하는 것이다. 그러나, 이 언어 습득 장치는 문법, 즉 의미 있는 문장을 만들어내는 패턴을 습득하는 데 도움을 줄 뿐, 언어의 전달 방식을 습득하게 해 주지는 않는다. 왜냐하면 인간의 언어는 소리를 통해서 전달하는 구어와, 손동작을 이용하는 수어 등 여러가지 전달 방식이 존재하기 때문이다.
        
    \section{문제 6번}
        
        임의적인 단어의 나열은 문법적으로 올바른 문장을 구성할 수 없다. (물론, 그 언어를 유창하게 할 수 있는 청자라면 단어의 의미로부터 화자의 의도를 유추할 수 있겠지만 그래도 문법적으로는 완전히 틀린 문장이다.) 인간의 언어를 유심히 관찰해 보면, 문장을 생성하는 규칙이 명확하게 정해져 있는 것처럼 보인다. 예를 들어, 한국어는 보통 주어-목적어-서술어 순으로 문장이 쓰이지만 영어는 보통 주어-서술어-목적어 순으로 문장이 전개된다. 뿐만 아니라, 주어에 들어가는 명사 또한 하나의 단어 뿐만이 아니라 명사 역할을 하는 하나의 구가 위치할 수도 있다. 이는 서술어와 목적어도 마찬가지이다. 따라서, 인간 언어는 여러 깊이의 계층으로 이루어져 있다고 볼 수 있다.
    
    \section{문제 7번}\
    
        \emph{보기} 문장을 나무 그림으로 표현하면 다음과 같다.
        
        \synttree{0}
        [S
            [NP
                [PP
                    [NP
                        [AP
                            [N [키가]]
                            [Adj [큰]]
                        ]
                        [N [형]]
                    ]
                    [P [의]]
                ]
                [N [친구가]]
            ]
            [VP [V [왔다]]]
        ]
        
        다음과 같은 나무 그림도 문법적으로 문제는 없다.
        
        \synttree{0}
        [S
            [NP
                [AP
                    [N [키가]]
                    [Adj [큰]]
                ]
                [NP
                    [PP
                        [N [형]]
                        [P [의]]
                    ]
                    [N [친구가]]
                ]
            ]
            [VP [V [왔다]]]
        ]

    \section{문제 8번}
    
        \emph{보기} (ㄱ)의 프랑스어 문장을 나무 그림으로 표현하면 다음과 같다.
        
        \synttree{0}
        [S
            [NP
                [Spec]
                [N' [N [Je]]]
            ]
            [VP
                [Spec]
                [V'
                    [V [mang\'{e}]]
                    [NP
                        [Spec [souvent]]
                        [N' [N [pommes]]]
                    ]
                ]
            ]
        ]
        
        한편, \emph{보기} (ㄴ)의 영어 문장을 나무 그림으로 표현하면 다음과 같다.
        
        \synttree{0}
        [S
            [NP
                [Spec]
                [N' [N [She]]]
            ]
            [VP
                [Spec]
                [V'
                    [Adjunct [often]]
                    [V'
                        [V [eats]]
                            [NP
                                [Spec]
                                [N' [N [apples]]]
                            ]
                    ]
                ]
            ]
        ]
        
        위 나무 그림에서 알 수 있듯, 프랑스어 문장에는 "종종"을 뜻하는 부사 souvent가 명사의 Specifier인 것으로 분석할 수 있다. souvent가 명사의 의미를 한정하지 않는 것이 명백하기에 X' 이론이 이 경우 맞지 않는다고 볼 수 있다. 이 문제는 souvent라는 부사가 문법적으로 위치해야 할 곳에서 현재 위치로 이동했다고 생각함으로서 해결할 수 있다.
    
    \section{문제 9번}
    
        발화 산출 시의 "운동 지시"와 발화 지각 시의 "음향 정보"는 호환되지 않는 형식이다. 먼저, 기본적으로 정보의 종류가 다르다. "운동 지시"는 뇌에서부터 입 주위의 근육에 전달되는 전기적 정보로, 근육을 움직이도록 지시한다. 반면, "음향 정보"는 귀 안의 청신경으로부터 들어오는 '음파', 또는 '진동'에 관한 정보로, 이 정보만으로 뇌 이외의 다른 부위에 직접적으로 영향을 줄 수는 없다. 그러나 분명 보통 사람은 들려오는 소리를 입으로 비슷하게 따라 내는 것을 일상적으로 하고 있다. (다른 사람의 말을 따라 하는 것처럼 언어적일 수도, 새가 지저귀는 것을 흉내 내는 것처럼 비언어적일 수도 있다.) 음향 정보와 운동 지시는 전혀 다른 형식의 정보인데도 소리를 흉내 낼 수 있는 이유는 인간의 뇌에서 음향 정보를 운동 지시로 번역하는 작업을 하고 있기 때문이다. 이 작업은 인간이 손을 움직일 때 별 생각 없이 자연스럽게 움직일 수 있는 것과 같이 선천적으로 타고난 인간의 능력이라고 할 수 있다.
    
    \section{문제 10번}
    
        물론 언어학에 관한 이론을 공부하는 것도 즐겁지만, 실제로 배운 이론을 실제 언어에 적용해볼 수 있는 기회가 더 있었으면 좋겠습니다. 만약 과제를 낼 예정이 있으시다면, 사람들에게 익숙한 언어 말고도 조금 생소한 언어에 대해 제가 배운 내용을 적용해 볼 수 있는 과제가 있으면 조금 더 재미있게 언어학을 공부할 수 있게 될 것 같습니다.
        
\end{document}
