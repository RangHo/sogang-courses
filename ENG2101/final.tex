\documentclass{mla}

\firstname{Juhun}
\lastname{Lee}
\instructor{Professor Jayoung Jeon}
\course{Writing about English Literature}
\date{21 June 2023}

\addbibresource{final.bib}

\title{Lifelong Friendships Start with an IP Address}

\begin{document}

It is often stated that the world is shrinking as time progresses.
Advancements in modern technology have significantly enhanced the speed and accessibility of communication with individuals in remote locations.
In the present day, the internet enables people to engage in real-time conversations with individuals from any location, granted that they have a stable internet connection.
This mode of communication has given rise to the emergence of a novel concept known as \textit{virtual reality}.

The concept of virtual reality has been extensively explored in various publications, including science fiction and plays.
However, one play stands out for its intriguing interpretation of the virtual world.
\textit{Water by the Spoonful}, by Quiara Alegr\'ia Hudes, is a second installment of the ``Elliot Trilogy'', a series of plays that tell the story of a Iraq War veteran.
The play unfolds in two distinct realms: the tangible real world, depicted through physical chairs, and the interconnected online world, depicted as the ``empty space connecting the chairs'' \autocite[3]{Hudes_2017}.
Hudes's \textit{Water by the Spoonful} illustrates the potentials and limitations of the online space, highlighting its profound impact on the characters of the play while hinting its temporary nature.
Through her portrayal, Hudes emphasizes the transformative power of virtual connections while also exploring its inherent limitations and and complexities of the online world.

While most agree that anonymity is one of the defining characteristics of the internet, it is important to establish a clear definition of the term.
According to Merriam-Webster dictionary,``anonymity'' is defined as ``the quality of lacking individuality, distinction, or recognizability'' \autocite{Merriam-Webster}.
However, it is worth noting that internet users do not entirely lose their recognizability.
For instance, when Orangutan returns to the chat room after three months of inactivity, Haikumom greets her, exclaiming ``Orangutan! Jesus, I thought my primate friend had disappeared back to the jungle'' \autocite[13]{Hudes_2017}.
Haikumom's immediate reaction to Orangutan's return contradicts the literal definition of anonymity.
In fact, scholars have categorized the concept of anonymity into two: technical anonymity and social anonymity.
When discussing anonymity in the context of the internet, it typically refers to the social aspect, where users perceive themselves to be deindividualized and unidentifiable \autocite{Hayne_Rice_1997}.
In other words, when using the internet, users tend to dissociate themselves from their real-world identity and construct a new persona.

Haikumom and Odessa Ortiz, two characters in \textit{Water by the Spoonful}, exemplify this phenomenon.
Haikumom portrays a caring and nurturing attitude, serving as the mother figure within the chat room, while Odessa embodies an indifferent and irresponsible attitude as Elliot's birthmother.
Believing that the chat room provides an anonymous space for former drug addicts to connect, Odessa takes on the persona of Haikumom, adopting a new identity within the chat room.
This phenomenon of adopting a new identity within the chat room is not unique to Odessa, but extends to other members of the chat room.
However, the disparity between their online personas and real identities is not as distict as it is in Odessa's case.

These newly adopted identities are not sustainable in the long term.
Once a user logs off from the chat room---that is, breaks their anonymity to the real identity---the user must revert back to their real identity.
One instance that illustrates this re-assumption occurs when Elliot and Yaz interrupt the conversation between Haikumom and Fountainhead.
When Elliot goes on a rant, exposing Odessa's past as ``Mami Odessa'', Odessa keeps silent until Fountainhead leaves the caf\'e in discomfort \autocite[55]{Hudes_2017}.
With Elliot's revelation, the illusion of anonymity within the chat room is shattered for Odessa.
Consequently, Odessa can no longer uphold her role as Haikumom, the commpassionate counselor for vulnerable Fountainhead.

Although these internet personas have a temporary nature, they should not be dismissed as mere fabrications.
According to a study conducted by K. M. CHristopherson in 2007, assuming a different online identity without negative social consequences ``may allow for an individual to take a different perspective on life and change their behavior for the better'' \autocite{Christopherson_2007}.
Despite individuals identifying themselves as distinct entities from their real-life selves, it is important to recognize that they remain the same physical person.
Consequently, the experiences gained through the internet persona are preserved and indirectly influence the real-life identity, and reciprocally so.

Furthermore, Hudes suggests that the online world plays a significant role when intertwined with the real world.
According to Leslie Durham's book \textit{Women's voices on American Stages in the Early Twenty-First Century}, Hudes's innovation lies in her approach to stage space, which implies ``activity and communication in one space will have measurable effect, if not governance, in the other'' \autocite[117--118]{Durham_2013}.
Hudes specifically directs the actors to refrain from miming typing when portraying conversations that take place in the online world \autocite[3]{Hudes_2017}.
Through her deliberate choice to make conversations in the chat room and real life indistinguishable, Hudes conveys the notion that actions in the online world hold equal significance and importance as those in real life, highlighting the reciprocal influence between two spaces.

An example that exemplifies how an online identity and the real-life identitity influence each other is the character of Fountainhead.
Fountainhead, whose real identity is John, is a once-wealthy programmer and businessman.
During the play, he becomes a member of Haikumom's drug addiction recovery website.
Initially, other members of the website react negatively to him as he downplays his drug addiction, considering it a minor concern in his life.

Fountainhead serves as an embodiment of John's lingering pride.
Upon his initial introduction ot the chat room, Fountainhead consistently highlights his achievements, including his successful programming company and his family \autocite[23--24]{Hudes_2017}.
Fountainhead deeply values his accomplishments and regards his ongoing struggle with a cocaine addiction, along with the resulting troubles, as a source of shame that must be hidden.
As a result, although John's true self consciously chooses to join a drug addiction recovery group, his pride conflicts with this decision, leading to the formation of his alternate identity as Fountainhead.

However, Fountainhead's attitude towards addiction undergoes a significant shift as the play progresses.
During Fountainhead's second login to the chat room, Chutes\&Ladders intentionally provokes him: ``If you're not a crackhead, leave, we don't want you, you are irrelevant, get off my lawn, go'' \autocite[41]{Hudes_2017}.
Although Chutes\&Ladders's attitude appears highly hostile towards Fountainhead, his intention is for hom to confront his pride, acknowledge his acddiction, and allow other members in the chat room to offer support for his recovery.`
Fountainhead becomes aware of this underlying intention and gradually becomes more receptive to the other members in the chat room.
He goes beyond his internet persona by personally seeking help from Haikumom and assisting here when shie is hospitalized after an overdose incidient.
This change in behavior clearly demonstrates the lasting positive impact that stems from an initially virtual relationship.

The significant change in Fountainhead's behavior is noteworthy as it signifies a complete departure from his previous obession with being a successful entrepreneur.
Daniel Dufournaud argues in his Essay ``When things are bad: Entrepreneurial failure and Levinasian ethic in Quiara Alegr\'ia Hudes's \textit{Water by the Spoonful}'', ``to call oneself an addict is to collapse socioeconolmically mediated relationality into an ethical form of sociality'' \autocite[452]{Dufournaud_2020}.
In this regard, John relinquished not only his online persona as Fountainhead but also his entire proud past.
By fully accepting the reality of his drug addiction, he let go of his past, which ultimately led him to discover a path towards a brighter future.

The concept of virtual reality has transformed the way people communicate and interact with others, bridging the physical distance between individuals.
Hudes's play, \textit{Water by the Spoonful}, delves into the intricate dynamics of the online world, illustrating both its potentials and limitations.
Through the exploration of virtual connections, Hudes highlights the transformative power of these relationships while also addressing the complexities and temporary nature of the online space.
It is evident that the internet has the capacity to shape and influence its users' identities, as seen in the characters of Haikumom and Fountainhead.

Furthermore, the interplay between online identities and real-life identities is a significant aspect to consider.
While users may adopt new personas within the online realm, they ultimately remain the same physical individuals.
The experiences and interactions encountered through the virtual space have the potential to influence and shape their real-life identities, and vice versa.
\textit{Water by the Spoonful} exemplifies this relationship through the character of Fountainhead, who initially hides his drug addiction and struggles with pride but gradually undergoes a transformation within the online chat room.
Fountainhead's virtual interactions enable him to confront his pride, seek support, and ultimately change his behavior for the better. This demonstrates the lasting impact that can arise from virtual relationships and highlights the interconnected nature of online and offline identities.

By examining the characters' interactions and the evolution of their identities in \textit{Water by the Spoonful}, it is evident that virutal connections, although temporary, can have a profound impact on individuals' lives.
The interplay between online and offline identities further emphasizes the significance of these virtual relationships.
As the society continues to navigate the ever-expanding virtual landscape, it is crucial to recognize the power of the online world and the ways in which it shapes perceptions, behaviors, and ultimately, identities.

\end{document}