\documentclass{mla}

\usepackage{xcolor}
\usepackage{siunitx}

\firstname{Juhun}
\lastname{Lee}
\instructor{Professor Lee, Hyun Sook}
\course{TLS1004--02}
\date{17 March 2024}

\title{Cats As Superior Pets, in the Modern World}

\addbibresource{paper.bib}

\begin{document}
\indent
\color{red}
There is a question that \textit{always} causes an online flame war: do cats make better pets than dogs?
\color{black}
The internet is filled with lovely videos of cats and dogs---yet, it seems the public cannot reach a consensus when determining the ``best pet''.
This question is especially important as many modern, urban residents suffer a lack of companionship.
While the world becomes increasingly ``smaller''---that is, it is much easier to communicate with others than before---due to social and economic complications, many young residents fail to find a partner.
As an alternative, many seek to adopt a pet in order to satisfy their emotional needs.
However, it is extremely important to pick the right animal when getting a pet.
Provided that the soon-to-be owner does not have any medical conditions against them,
\color{red}
cats are more suitable in a modern environment.

\color{green}
The economic advantage of owning a cat is a significant reason why cats are better for young people looking for companions.
\color{blue}
In a crowded urban environment like Seoul, young people, with limited budget to spend on housing, often live in studio apartments dubbed as the ``One-rooms''.
In Seoul, the average size of studio apartments is about \SI{19.8}{\square\m}.
Dogs mostly reside on the floor and possibly on the bed, and \SI{20}{\square\m} is not nearly enough space for most breed of dogs.
However, cats can, if not love to, climb onto high places in a room.
Therefore, cat owners can also utilize the vertical space in their small studio apartments, effectively multiplying the room for their pets.
This eliminates the need for a spacious room, which is a huge demand for a city like Seoul where affordable homes are not available.
Furthermore, the general care for cats cost about 15\% less than dogs.
According to a survery in the United States, the lifetime cost of a cat ranges from \$13,625 to \$17,510, where dogs cost from \$16,607 to \$22,423.
As cats are territorial animals, owners do not have to---or rather, should not---walk their cats.
Since the owners do not need extra walk-related equipments, cats carry less initial cost.
Throughout their lifetime, cats consume less calories than dogs, effectively reducing the recurring cost of pet food as well.
Cats are not only space-efficient pets, but also cost-efficient pets.

\color{green}
Cats' distinctive traits are also more attractive in an urban environment.
\color{blue}
Considering their ancestors, cats and dogs require different amount of attention and care.
As dogs formed packs before being domesticated, dogs require the owner to be around them constantly.
On the other hand, cats are solitary animals---while they do require human companionship, they handle inevitable situations where they need to stay alone much better than dogs do.
Although the modern technological advancements have allowed owners to stay with their pets at all times, cats provide more flexibility in the schedule for humans.
Furthermore, cats generally outlive dogs by a significant margin.
On average, cats live around fifteen to twenty years, where dogs live around five to ten years shorter.
Additionally, the self-defense capabilities of cats are far greater than those of dogs, thanks to their flexibility, agility, and short response time.
Their survival instinct is so great that it has been preserved in a proverb---``cats have nine lives''.
Regardless of where they live, cats lower the risk of tragic experiences that inevitably comes with owning a pet.

\color{black}
It is time to settle the debate of cats versus dogs, once and for all.
Considering their economic advantages and behavioral traits, cats are superior to dogs, at least in a crowded city.
While it may sound unethical to some, the cost of having pets and their behaviors are extremely important.
Without a careful consideration, owning a pet would be disastrous, both for the owner and the pet.
Therefore, those who are looking for a mental healtcare from a cuddly pet should consider a cat first.
\end{document}
