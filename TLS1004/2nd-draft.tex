\documentclass{mla}

\usepackage{siunitx}

\firstname{Juhun}
\lastname{Lee}
\instructor{Professor Lee, Hyun Sook}
\course{TLS1004--02}
\date{7 April 2024}

\addbibresource{paper.bib}

\title{Cats: The More Suitable Pets for the Modern Society}

\begin{document}
There is a question that \textit{always} causes an online flame war: do cats make better pets than dogs?
The internet is filled with lovely videos of cats and dogs---yet, it seems the public cannot reach a consensus when determining the ``best pet''.
This question is especially important as many modern, urban residents suffer a lack of companionship.
Forming personal relationship today is much easier than in the past thanks to the advent of modern technologies
However, ironically, due to social and economic complications, many young residents fail to find their partners. 
As an alternative, many seek to adopt a pet in order to satisfy their emotional needs.
However, it is extremely important to pick the right animal when getting a pet.
In the increasingly crowded modern society, cats make more suitable pets for citizens.

The economic advantage of owning a cat is a significant reason why cats are better for young people looking for companions.
In a crowded urban environment like Seoul, young people, with limited budget to spend on housing, often live in studio apartments dubbed as the ``One-rooms''.
In Seoul, the average size of studio apartments is about \SI{19.8}{\square\m}.
Dogs mostly reside on the floor and possibly on the bed, and \SI{20}{\square\m} is not nearly enough space for most breed of dogs.
However, cats can, if not love to, climb onto high places in a room.
Therefore, cat owners can also utilize the vertical space in their small studio apartments, effectively multiplying the room for their pets.
This eliminates the need for a spacious room, which is a huge demand for a city like Seoul where affordable homes are not available.
Furthermore, the general care for cats cost about 15\% less than dogs.
According to a survey in the United States, the lifetime cost of a cat ranges from \$13,625 to \$17,510, where dogs cost from \$16,607 to \$22,423.
As cats are territorial animals, owners do not have to---or rather, should not---walk their cats.
Since the owners do not need extra walk-related equipments, cats carry less initial cost.
Throughout their lifetime, cats consume less calories than dogs, effectively reducing the recurring cost of pet food as well.
Cats are not only space-efficient pets, but also cost-efficient pets.

Cats' distinctive traits are also more attractive in an urban environment, where the owner frequently has to leave pets by themselves.
Considering their ancestors, cats and dogs require different amount of attention and care.
As dogs formed packs before being domesticated, dogs require the owner to be around them constantly.
On the other hand, cats are solitary animals---while they do require human companionship, they handle inevitable situations where they need to stay alone much better than dogs do.
Additionally, cats have strong self-defense instincts, which lower the risk of injury when left unattended.
Cats and dogs have much faster reflexes than humans---sub-\SI{100}{\ms} compared to around \SI{250}{\ms}, respectively.
However, the reflex of cats can be as low as \SI{20}{\ms} whereas dogs average at around \SI{80}{\ms}.
Combined with the superb agility of cats, their fast reflexes allow cats to escape from dangers that may occur in a house---an improperly stored dish falling from the cupboard, for example.
In a modern society, often the owners must leave home unexpectedly, and it is extremely important to leave pets alone without worrying their emotional and physical well-being.
As such, cats provide much greater flexibility in the schedule for humans.

In conclusion, cats are more appropriate pets for busy, modern society, especially for those who live in crowded city like Seoul.
In general, owning a cat is more economically viable than owning a dog.
While one may not have enough space for a dog, the same space may be enough for a cat.
With the additional benefit of a lower support cost of a cat than that of a dog, it is clear that cats are more affordable pet for people nowadays.
Also, cats feel less stress than dogs do when left alone at home, due to the solitary, territorial nature of their ancestors.
Furthermore, their fast reflexes and incredible agility keep them safe when inevitable accidents happen when owners are away.
These traits of a cat allows the owners to leave their homes without worries, allowing for more efficient use of their time.
Therefore, those who are looking for a mental healtcare from a cuddly pet should consider a cat first.
\end{document}
