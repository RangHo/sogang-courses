\documentclass{translation}

\usepackage{hyperref}

\firstname{Juhun}
\lastname{Lee}
\instructor{Professor Hyejin Kim}
\course{Translation Practicum}
\date{5 June 2024}

% The Lottery
\title{뽑기 Part 2}

\begin{document}
% Mr. Martin and his oldest son, Baxter, held the black box securely on the stool until Mr. Summers had stirred the papers thoroughly with his hand.
서머스 씨가 손을 넣어 종이를 남김없이 휘젓는 동안 마틴 씨와 장남 백스터는 상자를 의자 위에 단단히 고정해 두었다.
% Because so much of the ritual had been forgotten or discarded, Mr. Summers had been successful in having slips of paper substituted for the chips of wood that had been used for generations.
이미 의례 절차의 상당수가 버려지고 잊혀진 터라, 서머스 씨는 몇 세대에 걸쳐 내려오는 전통 나무 제비 대신 종이 제비를 쓰자고 최근 사람들을 설득하는 데 성공했다.
% Chips of wood, Mr. Summers had argued, had been all very well when the village was tiny, but now that the population was more than three hundred and likely to keep on growing, it was necessary to use something that would fit more easily into the black box.
마을이 작던 시절에는 얼마든지 나무 조각을 쓸 수 있었지만 이제는 마을 사람이 삼백이나 되고 줄어들 기미도 보이지 않으니 검은 상자에 집어넣기 쉬운 무언가로 바꾸어야 한다는 것이다.
% The night before the lottery, Mr. Summers and Mr. Graves made up the slips of paper and put them into the box, and it was then taken to the safe of Mr. Summers’ coal company and locked up until Mr. Summers was ready to take it to the square next morning.
뽑기 전날 밤, 서머스 씨와 그레이브스 씨는 종이 조각을 미리 만들어 상자 안에 넣어 두고, 상자는 다음날 아침 광장에 가기 전까지 서머스 씨의 석탄 회사 금고에 단단히 잠겨 보관해 두었다.
% The rest of the year, the box was put away, sometimes one place, sometimes another; it had spent one year in Mr. Graves’ barn and another year underfoot in the post office, and sometimes it was set on a shelf in the Martin grocery and left there.
나머지 363일 동안 상자는 이곳저곳에서 보관해왔다.
한 해는 그레이브스 씨의 마굿간에서, 한 해는 우체국 발치에 있었고, 또 한 해는 마틴 잡화점 선반 한 켠에 방치되기도 했다.
% 

% There was a great deal of fussing to be done before Mr. Summers declared the lottery open.
뽑기 개시를 선언하기 전에 서머스 씨가 처리해야 할 귀찮은 일도 많았다.
% There were the lists to make up—of heads of families, heads of households in each family, members of each household in each family.
미리 집안마다 집안 대표, 집안 내 가정마다 가정 대표와 가정 구성원들을 모두 조사해서 명단을 만들어야 했다.
% There was the proper swearing-in of Mr. Summers by the postmaster, as the official of the lottery; at one time, some people remembered, there had been a recital of some sort, performed by the official of the lottery, a perfunctory, tuneless chant that had been rattled off duly each year; some people believed that the official of the lottery used to stand just so when he said or sang it, others believed that he was supposed to walk among the people, but years and years ago this part of the ritual had been allowed to lapse.
우체국장으로부터 절차에 맞게 뽑기 집행관 취임식도 치뤄야 했다.
누군가가 기억하기를, 원래는 집행관이 평탄한 음조로 기도문을 줄줄이 외는 암송회같은 것이 매년 있었다는 모양이다.
그 중에서도 암송을 할 때에 집행관이 혼자 앞에 서 있었다고 믿는 사람들과, 인파 사이로 걸어다니며 기도문을 외웠다고 믿는 사람들로 갈렸는데, 어찌되었건 오래 전에 이런 예식은 세월의 풍파에 사라졌다. 
% There had been, also, a ritual salute, which the official of the lottery had had to use in addressing each person who came up to draw from the box, but this also had changed with time, until now it was felt necessary only for the official to speak to each person approaching.
또, 집행관은 제비를 뽑으러 오는 사람 하나하나에게 특별한 인사 의례를 거행해야 했지만, 시간이 지나면서 이것 또한 바뀌어 이제는 집행관이 참가자에게 말을 건네는 것으로 바뀌었다.
% Mr. Summers was very good at all this; in his clean white shirt and blue jeans, with one hand resting carelessly on the black box, he seemed very proper and important as he talked interminably to Mr. Graves and the Martins.
서머스 씨는 이 일에 적임이었다.
새하얗고 깔끔한 셔츠에 청바지를 입고 한 손은 무심한 듯 검은 상자에 올린 채 그레이브스 씨와 마틴 일가에게 끝없이 말을 건네는 그 모습은 무척이나 올바르고 중요한 인물처럼 보였다.
% 

% Just as Mr. Summers finally left off talking and turned to the assembled villagers, Mrs. Hutchinson came hurriedly along the path to the square, her sweater thrown over her shoulders, and slid into place in the back of the crowd.
서머스 씨가 겨우 말을 끝맺고 마을 사람들을 향해 돌아보자마자, 허친슨 부인이 스웨터를 어께에 걸치고 허겁지겁 광장 도보를 달려와 사람들 뒤로 미끄러지듯 들어왔다.
% “Clean forgot what day it was,” she said to Mrs. Delacroix, who stood next to her, and they both laughed softly.
허친슨 부인은 옆자리에 선 들라크루아 부인에게 변명을 늘어놓았다.\par
``오늘이 무슨 날인지 까맣게 잊었지 뭐예요.''\par
두 부인은 나지막이 웃었다.
% “Thought my old man was out back stacking wood,” Mrs. Hutchinson went on, “and then I looked out the window and the kids was gone, and then I remembered it was the twenty-seventh and came a-running.”
허친슨 부인은 말을 이었다.\par
``남편이 뒷마당에서 나무를 하고 있는 줄 알았는데 창문을 보니 아이들이 없는 거예요. 달력을 보니 오늘이 27일이길래 바로 뛰어나왔죠.''\par
% She dried her hands on her apron, and Mrs. Delacroix said, “You’re in time, though. They’re still talking away up there.” 
``안 늦었으니 괜찮아요. 아직 저기서 이야기하는 중이에요.''\par
허친슨 부인이 앞치마에 손을 닦는 사이에 들라크루아 부인이 말했다.
% 

% Mrs. Hutchinson craned her neck to see through the crowd and found her husband and children standing near the front.
허친슨 부인은 인파 너머로 목을 쭉 뺐고 곧 맨 앞자리 근처에 서있는 남편과 아이들을 발견했다.
% She tapped Mrs. Delacroix on the arm as a farewell and began to make her way through the crowd.
들라크루아 부인의 어깨를 두드려 작별인사를 건네고 허친슨 부인은 인파를 헤쳐 나아갔다.
% The people separated good-humoredly to let her through; two or three people said, in voices just loud enough to be heard across the crowd, “Here comes your Mrs., Hutchinson,” and “Bill, she made it after all.”
사람들은 짓궂게 놀리면서도 허친슨 부인에게 길을 터 주었다.
그 중 두세 명은 딱 모두가 들을 수 있는 소리로 ``드디어 자네 부인이 오는구만, 허친슨'', ``빌, 부인이 오기는 했네.'' 하고 농담을 했다.
% Mrs. Hutchinson reached her husband, and Mr. Summers, who had been waiting, said cheerfully, “Thought we were going to have to get on without you, Tessie.”
허친슨 부인이 남편 곁에 다가가자, 서머스 씨는 즐거운 듯이 말했다.\par
``테시, 당신 없이 시작하려던 차였습니다.''\par
% Mrs. Hutchinson said, grinning, “Wouldn’t have me leave m’dishes in the sink, now, would you, Joe?,” and soft laughter ran through the crowd as the people stirred back into position after Mrs. Hutchinson’s arrival.
``설거지거리를 싱크대에 그대로 두고 나오라니, 당신이 그럴 리가 없잖아요. 안 그래요, 조?''\par
허친슨 부인은 생긋 웃어 보였다.
% 


% “Well, now,” Mr. Summers said soberly, “guess we better get started, get this over with, so’s we can go back to work. Anybody ain’t here?”
``좋습니다.''\par
서머스 씨가 냉정하게 소리를 높였다.\par
``이제 시작해야 할 것 같군요. 빨리 끝내버리고 일하러 돌아갑시다. 여기 안 계신 분 있습니까?''
% 

% “Dunbar,” several people said.
``던바요.'' 하고 누군가가 소리쳤다.
% “Dunbar, Dunbar.” 
그러자 ``던바요, 던바가 없소.'' 하고 몇몇이 거들었다.
% 

% Mr. Summers consulted his list.
서머스 씨는 명단을 훑어보고는 말했다.\par
% “Clyde Dunbar,” he said.
% “That’s right. He’s broke his leg, hasn’t he? Who’s drawing for him?” 
``클라이드 던바 씨로군요. 다리가 부러졌다고 했던 것 같은데, 던바 씨 대신 뽑는 사람은 누구입니까?''
\end{document}
