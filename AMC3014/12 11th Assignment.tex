\documentclass{translation}

\usepackage{hyperref}

\firstname{Juhun}
\lastname{Lee}
\instructor{Professor Hyejin Kim}
\course{Translation Practicum}
\date{29 May 2024}

% The Lottery
\title{뽑기}

\begin{document}
% The morning of June 27th was clear and sunny, with the fresh warmth of a full-summer day; the flowers were blossoming profusely and the grass was richly green.
6월 27일 아침은 맑고 햇볕이 내려쬐는 상쾌한 한여름날이었다.
꽂은 한가득 피어났고, 잔디는 싱그러운 푸른색을 뽐냈다.
% The people of the village began to gather in the square, between the post office and the bank, around ten o’clock; in some towns there were so many people that the lottery took two days and had to be started on June 26th, but in this village, where there were only about three hundred people, the whole lottery took only about two hours, so it could begin at ten o’clock in the morning and still be through in time to allow the villagers to get home for noon dinner.
열 시 즈음이 되자, 마을 사람들이 하나둘씩 우체국과 은행 사이 광장에 모여들기 시작했다.
어떤 마을은 사람이 너무 많아서 6월 26일부터 꼬박 이틀을 뽑기에 쓴다는 모양이지만, 인구 300명 남짓의 이 마을에서는 두 시간이면 끝났다.
열 시에 뽑기를 시작하면 열두 시에 모두 집에서 점심을 양껏 먹을 시간이 남을 터였다.
% 

% The children assembled first, of course.
당연히 먼저 모인 것은 아이들이었다.
% School was recently over for the summer, and the feeling of liberty sat uneasily on most of them; they tended to gather together quietly for a while before they broke into boisterous play, and their talk was still of the classroom and the teacher, of books and reprimands.
얼마 전 여름방학을 맞고 어색한 해방감을 만끽하던 아이들은 쭈볏거리며 잠시 조용히 모여 있다가도 부산스럽게 놀기 시작하곤 했다.
그래도 아직 이야기 주제는 학교 교실과 선생님 이야기였고, 읽어야 했던 책과 들었던 잔소리였다.
% Bobby Martin had already stuffed his pockets full of stones, and the other boys soon followed his example, selecting the smoothest and roundest stones; Bobby and Harry Jones and Dickie Delacroix—the villagers pronounced this name “Dellacroy”—eventually made a great pile of stones in one corner of the square and guarded it against the raids of the other boys.
바비 마틴은 일찌감치 주머니에 돌을 가득 채워넣었고, 다른 남자아이들도 바비를 따라 동그랗고 매끈한 돌을 찾아 주머니에 넣었다.
바비와 해리 존스와 디키 델라크로와(마을 사람들은 다들 델라크로이라고 불렀다)는 광장 한구석에 커다란 돌무더기를 만들고선 다른 남자아이들이 훔치지 못하게 그 앞을 지키고 섰다.
% The girls stood aside, talking among themselves, looking over their shoulders at the boys, and the very small children rolled in the dust or clung to the hands of their older brothers or sisters.
여자아이들은 어깨너머로 남자아이들을 흘깃거리며 저들끼리 재잘거렸고, 어린 아이들은 먼지밭에서 뒹굴거나 언니오빠 손을 꼭 붙잡고 있었다.
% 

% Soon the men began to gather, surveying their own children, speaking of planting and rain, tractors and taxes.
곧 남자들이 모여 아이들을 살피면서 서로 농사일과 강우량을 걱정하고, 트랙터와 세금에 관해 이야기하기 시작했다.
% They stood together, away from the pile of stones in the corner, and their jokes were quiet and they smiled rather than laughed.
그들은 돌무더기에서 멀찍이 떨어져 서서 소리 내어 웃기보다는 빙긋 미소를 지으며 농담을 주고받았다.
% The women, wearing faded house dresses and sweaters, came shortly after their menfolk.
빛바랜 실내복과 스웨터를 입은 여자들은 잠시 뒤 남편들을 따라 도착했다.
% They greeted one another and exchanged bits of gossip as they went to join their husbands.
각자 남편이 서 있는 곳으로 향하며 여자들은 서로 인사를 나누고 가십거리를 수군거렸다.
% Soon the women, standing by their husbands, began to call to their children, and the children came reluctantly, having to be called four or five times.
남편 곁에 서자 여자들은 큰 소리로 아이들을 불렀다.
아이들은 이름을 네다섯 번 부르기 전까지는 들은 체도 않다가 슬금슬금 부모님에게 걸어왔다.
% Bobby Martin ducked under his mother’s grasping hand and ran, laughing, back to the pile of stones.
바비 마틴은 손을 뻗으려는 어머니를 피해 고개를 푹 숙이곤 깔깔거리며 돌무더기 옆으로 도로 달려갔다.
% His father spoke up sharply, and Bobby came quickly and took his place between his father and his oldest brother.
아버지가 언성을 높이자 바비는 곧장 돌아와 아버지와 큰형 사이에 자리를 잡았다.
% 

% The lottery was conducted—as were the square dances, the teen-age club, the Halloween program—by Mr. Summers, who had time and energy to devote to civic activities.
뽑기는 광장 무도회나 청년 클럽, 할로윈 행사와 똑같이 사회 활동에 참가할 시간과 여력이 있는 서머스 씨가 주도했다.
% He was a round-faced, jovial man and he ran the coal business, and people were sorry for him, because he had no children and his wife was a scold.
서머스 씨는 동그란 얼굴에 쾌활한 성격을 가진 남자로, 석탄 사업을 하는 사람이었다.
그럼에도 사람들은 자식이 없는데다 항상 아내의 쨍쨍거림을 견뎌야 하는 서머스 씨를 측은하게 여겼다.
% When he arrived in the square, carrying the black wooden box, there was a murmur of conversation among the villagers, and he waved and called, “Little late today, folks.”
그가 검정 상자를 가지고 광장에 도착하자 이미 사람들은 서로 웅성이며 대화를 나누고 있었다.
서머스 씨는 손을 흔들어 보이며 ``조금 늦었군요, 여러분.'' 하고 사람들을 불렀다.
% The postmaster, Mr. Graves, followed him, carrying a three-legged stool, and the stool was put in the center of the square and Mr. Summers set the black box down on it.
우체국장 그레이브스 씨가 서머스 씨를 따라 세 발 의자를 가져와 광장 한가운데에 두었다.
서머스 씨는 그 위에 검정 상자를 올려두었다.
% The villagers kept their distance, leaving a space between themselves and the stool, and when Mr. Summers said, “Some of you fellows want to give me a hand?,” there was a hesitation before two men, Mr. Martin and his oldest son, Baxter, came forward to hold the box steady on the stool while Mr. Summers stirred up the papers inside it.
서머스 씨가 상자로부터 멀찍이 떨어져 선 마을 사람들을 향해 ``손 좀 빌려주실 분 없습니까?'' 하고 물었다.
머뭇이는 마을 사람들 사이로 마틴 씨와 장남 백스터가 나와서 의자 위에 검정 상자를 단단히 붙들었다.
서머스 씨는 상자 안에 손을 집어넣어 종이조각을 휘저어 섞었다.
% 

% The original paraphernalia for the lottery had been lost long ago, and the black box now resting on the stool had been put into use even before Old Man Warner, the oldest man in town, was born.
원래 쓰이던 뽑기 상자는 잃어버린지 오래고, 당장 의자에 놓인 검은색 상자는 마을 최고령인 워너 영감이 태어나기 전부터 쓰던 물건이다.
% Mr. Summers spoke frequently to the villagers about making a new box, but no one liked to upset even as much tradition as was represented by the black box.
서머스 씨는 몇 번이나 상자를 새로 만들자고 마을 사람들을 설득했지만, 그 누구도 유구한 전통을 가진 검정 상자를 건드리려는 생각을 하지 않았다.
% There was a story that the present box had been made with some pieces of the box that had preceded it, the one that had been constructed when the first people settled down to make a village here.
지금 쓰는 상자를 만들 때 이 마을을 개척한 이들이 가져온 상자 조각을 집어넣었다는 이야기도 나돌았다.
% Every year, after the lottery, Mr. Summers began talking again about a new box, but every year the subject was allowed to fade off without anything’s being done.
매년 뽑기가 끝나면 서머스 씨는 상자를 새로 만들자고 이야기를 꺼내지만, 결국 매년 아무것도 하지 못한 채 흐지부지 넘어갔다.
% The black box grew shabbier each year; by now it was no longer completely black but splintered badly along one side to show the original wood color, and in some places faded or stained.
상자는 해가 갈수록 허름해져 이제는 완전한 ``검정'' 상자가 아니었다.
모서리는 쪼개져서 나무 색이 드러났으며 어떤 면은 물감이 흐릿해지고 어떤 면은 다른 색으로 물들었다.
\end{document}
