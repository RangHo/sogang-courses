\documentclass{translation}

\usepackage{hyperref}

\firstname{Juhun}
\lastname{Lee}
\instructor{Professor Hyejin Kim}
\course{Translation Practicum}
\date{24 April 2024}

% A Taxi Driver Staple of South Korea Has Inspired a New Restaurant
% Kisa is the Lower East Side follow-up from the C as in Charlie team
\title{한국 택시 기사의 한 끼, 뉴욕의 레스토랑이 되다}

\begin{document}
% Having opened in Noho in 2022, the Korean hotspot, C as in Charlie, tells the story of who the owners are: Koreans who moved to Atlanta and grew up sipping sweet tea, and later soju, in the South.
지난 2022년, 뉴욕 맨해튼 노호 지구에 문을 연 명물 한국식 음식점 ``씨 애즈 인 찰리(C as in Charlie)''는 어린 시절 미국 조지아 주 애틀랜타로 건너와 달콤한 아이스티를, 커서는 소주를 홀짝이며 자란 주인장의 삶을 담아낸다.
% The restaurant serves playful spins on their Korean Southern heritage through dishes like gruyere grits with galbi jus and a dessert made to look like a bagel, earning them a Michelin Bib Gourmand last fall.
이 식당은 한국계 미국 남부 주민의 생활상을 보여주는 음식을 선보인다.
갈비 소스를 곁들인 그뤼예르 치즈를 넣은 그리츠\footnote{굵게 간 말린 옥수수 등을 맑은 소금물이나 우유에 넣어 끓인 죽과 비슷한 요리. 미국 남부 지방에서 주로 먹는다.}와 베이글 모양으로 빚은 디저트는 미슐랭 빕 구르망\footnote{미슐랭 가이드에서 추천하는 가게 중, 적당한 가격으로 특출나게 맛있는 음식을 맛볼 수 있는 음식점 명단.} 명단에 오르는 영예까지 가져다 주었다.
%

% Their second restaurant is a story about their homeland.
이들의 두 번째 음식점에서는 고향의 이야기를 담아낸다.
%

% Kisa, short for kisa sikdang (also spelled gisa sikdang), or “driver’s restaurant,” opens Saturday, April 20, at 205 Allen Street, at Houston Street, on the Lower East Side, an homage to Korean diners that emerged in the ‘80s, catering to taxi drivers.
``기사식당''의 앞 두 글자를 따 그대로 음차한 ``기사(Kisa)''는 오는 4월 20일, 맨해튼 로어이스트사이드 하우스턴가에서 영업을 개시한다.
80년대 택시기사를 위해 생겨난 한국의 식당을 기리는 의미에서다.
% Unlike C as in Charlie, where shots flow freely, this isn’t a drinks-first restaurant, though they’re part of the menu, but like most diners, it’s about a satisfying meal for a good value.
술잔이 오고가는 ``씨 애즈 인 찰리''에서와는 달리, ``기사''는 술을 마시는 주점이 아니다.
차림표에 술이 있긴 하지만, ``기사''는 맛있는 음식을 착한 가격에 파는 여느 식당과 다르지 않다.
%

% The follow-up restaurant – the only kisa-style diner in NYC – comes from C as in Charlie partners, David JoonWoo Yun and Steve JaeWoo Choi (who were born in Korea and moved to Atlanta when they were kids), joined by Yong Min (YK) Kim.
뉴욕 시에서 유일무이한 이 기사식당은 어린 시절 애틀랜타로 이민 온 데이비드 윤 씨(본명 윤준우), 스티브 최 씨(본명 최재우)와 김용민 씨가 함께 설립했다.
``씨 애즈 인 찰리''를 세운 콤비가 다시 모인 것이다.
% The 36-seat spot echoes kisas in Korea, with its vintage TVs, wall-mounted fans, Korean calendars, and a coin-slot coffee machine that dispenses (in this case, free) coffee for drivers on the go.
최대 36명까지 수용 가능한 이 식당은 원조 기사식당처럼 벽에 오래된 TV와 벽걸이 선풍기, 한국식 종이달력을 달았고 문가에 무료 믹스커피 자판기를 놓았다.
% It’s entirely for walk-ins.
또, 예약 역시 받지 않는다.
%

% “With Kisa, we want to celebrate casual Korean cuisine and bring an authentic Korean diner experience to NYC,” says Choi.
``평범한 한식의 맛을 기리고, 한국식 식당에서 느낄 수 있는 경험을 뉴욕에서도 느낄 수 있게 하고 싶었어요.''
스티브 최 씨는 이렇게 설명했다.
%

% Chef Simon Lee, formerly of Jua, is overseeing a baekban homestyle menu with orders served on trays.
``기사''의 은쟁반 백반 한 상은 ``주아''에서 일하던 주방장 시몬 리 씨가 담당한다.
% Meals include jeyuk (spicy pork), thin sliced and marinated in gochujang sauce; bulgogi, marinated in a sweet soy sauce; bori bibimbap with barley rice, yeolmu kimchi (radish stem), gosari (frenbrake), shitake mushroom, and balloon flower roots; and jingeo bokkeum (spicy squid), stir-fried in a gochujang base, with carrot and scallion.
상차림은 제육볶음, 불고기, 오징어볶음과 함께 보리밥에 열무김치, 고사리, 표고버섯, 도라지가 들어간 비빔밥으로 구성되어 있다.
% Sets are around $30.
백반 정식은 30달러(약 41,000원) 정도다.
%

% While coffee and tea are on offer — catering to drivers, as it were — it’s also a gathering spot for post-shift, so there is a brief menu of beer and soju, though there is no bar.
운행 도중 끼니를 해결하는 기사를 위해서는 커피와 차를 제공하지만, 하루 일과가 끝난 기사를 위해 간단한 맥주와 소주도 차림표에 올라와 있다.
%

% “We want to honor... a spirit of jeong,” says Yun.
데이비드 윤 씨는 한국의 정을 더 소중히 여기고 싶었다고 한다.
% “Beyond serving Korean cuisine, Kisa will be a fun gathering space where our guests can come together over Korean spirits, baekban platters, and a love for Korean culture.”
```기사'는 한식을 대접하기만 하는 식당을 넘어, 손님들이 모여서 한국의 증류주를 나누고 한국식 백반 정식을 먹으며 한국 문화에 대한 관심과 사랑을 키우는 만남의 장이 되었으면 해요.''
%

% South Korea has over 300,000 taxis, with perhaps half in Seoul.
한국 국내에서 운행 중인 택시 대수는 300,000을 넘는다. 이 중 절반이 서울에 밀집되어 있다.
% Between rides, this type of restaurant is a mainstay, allowing drivers and those in search of an inexpensive, nutritious meal to fuel up and take a break at all hours.
기사식당은 수많은 택시기사들이 운행 사이사이에 언제든지 값싸고 영양가 있는 끼니를 먹으며 쉴 수 있는 대들보 같은 존재다.
% With Kisa located down the street from another cabbie favorite, Punjabi Deli, it’s an auspicious time to focus on restaurants catering to transit workers, with shows like Keep the Meter Running, following New York taxi drivers’ favorite spots and drawing in millions of views on social media.
``기사''로부터 몇 블록 내려가면 택시기사들이 추천하는 또 다른 식당, ``푼자비 델리''가 있다.
뉴욕 시내 택시를 타고 기사들이 가장 사랑하는 명소를 탐방하는 ``미터기만 돌려주세요(Keep the Meter Running)'' 시리즈가 SNS(사회관계망서비스)에서 백만 뷰를 넘기는 것을 보면, 운송업 종사자를 위한 음식점이 대성할 시기인 듯 하다.
%

% “The gisa sikdang is to Seoul what the taco truck is to L.A.: Eat where the drivers eat, and you’ll get an unvarnished taste of working-class Seoul, far from the neon glow of tourist hot spots and towering phalanxes of gleaming skyscrapers,” the Los Angeles Times reported.
LA 타임즈는 ``서울의 기사식당은 곧 로스앤젤레스의 타코 트럭과 같다. 택시기사가 먹는 밥을 먹음으로서 관광객으로 넘쳐나고 고층 빌딩이 번쩍이는 서울에서 벗어나 서울의 노동자의 일상을 있는 그대로 느낄 수 있다''고 평했다.
% And like the decline of the yellow cab, there is some concern among fans in South Korea that the era of the kisa sikdang is on the wane.
반면에 국내의 택시 수가 줄면서 기사식당의 시대가 끝을 맞는 것은 아닐까 걱정하는 목소리도 들려오고 있다.
%

% While a standard kisa sikdang might be open closer to round-the-clock, Kisa’s opening hours will be Tuesday through Saturday from 5 to 11 p.m.
보통 기사식당은 24시간 가까이 여는 일이 많지만, ``기사''는 화요일부터 토요일까지 오후 5시부터 11시까지만 영업한다.
\end{document}
