\documentclass{translation}

\usepackage{hyperref}

\firstname{Juhun}
\lastname{Lee}
\instructor{Professor Hyejin Kim}
\course{Translation Practicum}
\date{24 April 2024}

% A Good Prospect | Mining Climate Anxiety for Profit
\title{탐광하기 좋은 날 | 기후 불안을 파서 금을 캐는 방법}

\begin{document}
% The mining industry has always provided an economic justification for the displacement and exploitation of people all over the world: the colonization of the Americas for gold, silver, iron, and copper; blood diamonds in West and Central Africa; child labor in cobalt mines in the DRC; and thousands of deaths linked to paramilitaries financed by multinational mining companies in Colombia.
채광 업계에서는 언제나 경제 발전을 명분 삼아 주민들을 강제로 이주시키고 착취해왔다.
금, 은, 철, 구리를 이유로 아메리카 대륙을 식민지로 삼은 이들이 있지 않은가?
서아프리카와 중앙아프리카에서 수입하는 피투성이 다이아몬드\footnote{블러드 다이아몬드(blood diamond)는 아프리카 분쟁지역에서 전쟁 자금 조달을 위해 생산되는 다이아몬드를 말한다.}는 또 어떤가?
콩고의 코발트 광산에서 곡괭이를 드는 이는 아이들이 아닌가?
콜롬비아에서는 거대한 채광 기업이 후원하는 준군사조직이 수천 명을 죽이지 않았는가?
% The move to renewable energy will likely expose the poorest people on the planet to more of the same from these fierce extractive forces.
이런 일을 벌인 이들이 이제는 재생 가능 에너지 전환을 명목으로 세계 최빈곤층을 착취할 것이다.
% A Nature study published late last year found that more than half of the materials needed for the green energy transformation are located on or near relatively undeveloped land where Indigenous and peasant populations live.
작년 『네이처』 지에 수록된 한 연구는 녹색 에너지 전환에 필요한 원자재의 반 이상이 아메리카 원주민과 소농민이 모여 사는 저개발 지역에 집중되어 있다는 결과를 내놓았다.
% As mines encroach on these communities, they will be removed from their homelands or forced to live with profound industrial pollution.
광산이 이러한 지역을 스멀스멀 잠식하면 이들은 결국 고향을 떠나거나, 환경 오염을 매일같이 견디며 살아갈 수밖에 없다.
%

% The focus at this year’s PDAC was primarily on the economic benefits for these communities.
올해 PDAC 행사에서는 광산업이 원주민들에게 가져다주는 경제효과에 집중했다.
% First Nations speakers attested that mining can be good for Indigenous communities.
퍼스트 네이션\footnote{퍼스트 네이션(First Nation)은 유럽인들이 아메리카 대륙에 도착하기 전 캐나다 지역에 정착한 원주민들이 이룬 국가 내지는 공동체이다.} 출신 연사들은 광산업이 원주민에게 좋은 소식이라고 역설했다.
% “Before the mine, we had nothing,” Donny McCallum told the crowd at a panel on Indigenous economic inclusion.
% McCallum is a member of the Marcel Colomb First Nation (MCFN), in Manitoba.
캐나다 마니토바 주에 있는 마르셀 콜롬(Marcel Colomb) 퍼스트 네이션 소속인 도니 맥캘럼 씨는 원주민과의 경제 협력을 다루는 공개 토론회에 참여했다.
곧 토론회 청중에게 그는 이야기를 시작했다.
``광산이 있기 전, 우리에게는 아무 것도 없었습니다.''
% In 2022, the nation signed joint ventures with two contracting companies designed to help secure employment for members at a nearby gold mine.
2022년, 마르셀 콜롬 퍼스트 네이션은 주민들이 근처 금광에서 일할 수 있도록 두 위탁 회사와 공동 투자 계약을 맺었다.
% “We want a piece of the pie,” McCallum explained.
``우리도 파이 한 조각은 가져가야죠.''
% The panel’s moderator, Christian Sinclair, who is Opaskwayak Cree, encouraged Canada’s First Nations to follow MCFN’s example and form Indigenous economic development corporations.
맥컬럼 씨의 설명이 끝나자, 오파스콰약 크리(Opaskwayak Cree) 퍼스트 네이션 출신 사회자 크리스티안 싱클레어 씨는 다른 퍼스트 네이션도 마르셀 콜롬 퍼스트 네이션을 본보기 삼아야 한다고 극찬했다.
% He pointed to the Southern Ute Tribe of southwestern Colorado, which sits atop a rich formation of coal-bed methane.
그러면서 미국 콜로라도 주 남서부에 있는 남부 우테족의 성과도 함께 언급했다.
% The tribe used these resources to build a three-billion-dollar organization.
남부 우테족은 부족 영토에서 뽑아낸 석탄층 메탄가스로 시장 가치 30억 달러짜리 기업을 일구어낸 경력이 있다.
%

% The only note of hesitation about tying the well-being of First Nations to mining industry profits was sounded by an older man with a long ponytail during the Q&A portion of this session.
광산업의 발전은 곧 퍼스트 네이션의 발전이라는 주장은 모두가 만장일치라 받아들이는 듯 했다.
단 한 명 빼고는 말이다.
% “The land cannot sustain making the most amount of money in the least amount of time,” he told the panelists.
``이렇게나 짧은 기간에 큰돈을 만드는 이 사업을 대지가 과연 언제까지 감당할 수 있겠습니까?''
질의응답 시간이 시작되자, 긴 머리를 묶어올린 한 나이 지긋한 남자가 토론회 패널을 향해 따져 물었다.
%

% I caught up with this man, Rick Cheechoo, later on, at a reception for PDAC’s Indigenous Program over mini bison potpies and wild rice salad.
나는 이 남자를 PDAC 토착민 지원 프로그램 피로연장에서 다시 만났다.
% He is a member of the Moose Cree First Nation.
% A large gold mine operates near his nation’s traditional territory, he told me.
무스 크리(Moose Cree) 퍼스트 네이션 출신 릭 치추라고 자신을 소개한 그는 들소 포트파이\footnote{고기와 야채를 넣어 끓인 수프를 그릇에 담아 파이 시트를 덮어 구워낸 파이의 한 종류.}와 줄풀 샐러드를 담으며 나에게 무스 크리 자치령 근처 커다란 금광 이야기를 해 주었다.
% There have been benefits — he described agreements between the mining company and the First Nation that provide jobs and healthcare — but a company’s drive for profit puts it at odds with some tribal needs, especially preserving their culture and treaty rights and ensuring that families displaced by industry are compensated.
취업이 보장되고, 건강 보험도 회사 측에서 들어주는 등 이득이 없는 것은 아니었지만, 기업은 이익을 좇아 종종 원주민들과 충돌했다.
원주민의 문화와 자주성를 존중하려는 노력도, 집을 떠나야 했던 가족에게 지급하는 보상금도 부족했다.
%

% Historically, mining has brought disruption and violence to Canada’s First Nations, and there’s no reason to believe that’s changing.
광산업은 캐나다의 여러 퍼스트 네이션에게 폭력을 행사하며 원주민들을 불안의 구렁텅이에 빠뜨려 왔다.
여태까지 그러하였듯, 앞으로도 변함없이 그러할 터다.
% The past decade has seen numerous confrontations, with protestors blockading mines and arrested en masse by militarized police.
지난 10년은 특히 충돌이 빈번했다.
시위대는 광산을 막으며 목소리를 높였고, 곧 무장 경찰이 대거 연행해 가는 일이 반복됐다.
% Even as the conference was happening, members of the Naskapi and Innu nations were fighting an iron mine in Quebec — a situation that, as far as I could tell, was not addressed by anyone at PDAC.
PDAC 컨퍼런스가 한창인 와중에도 나스카피(Naskapi) 퍼스트 네이션과 이누(Innu) 퍼스트 네이션은 캐나다 퀘벡 주에서 철광과 맞서 싸우고 있었다.
내가 지금까지 만난 그 누구도 이들의 목소리를 대변해주지 않았다.
%

% Are there no alternatives to this rush to extract the world’s metals and minerals?
지구상의 모든 금속과 희토류를 하나도 남김없이 캐내는 것 말고는 정말 대안이 없는 것일까?
% At PDAC, almost no one asks this question.
PDAC에서는 들어볼 수 없는 질문이다.
% The assumption was baked into virtually every convention-floor booth and conference-room panel: minerals must be extracted.
행사장 부스부터 토론회장 패널까지, PDAC의 모두가 동의하는 부분이 한 가지 있었다.
광물은 채굴되어야 한다는 것이다.
% The market and a cooler planet demand it.
시장과 펄펄 끓는 지구가 그리 하라고 명령했기 때문이다.
% ESG and other signs of virtuous consumption, like partnerships with Indigenous communities, permit it.
지속 가능한 발전을 위해서라면 그리 해도 된다고 허가했기 때문이다.
원주민과의 상생 등을 지향하는 윤리적 소비자가 그리 해도 좋다고 허락했기 때문이다.
% Even beyond PDAC, alternate visions can be hard to come by, but a new study from the Climate and Community Project and the University of California, Davis aims to expand our imaginations.
비단 PDAC 뿐만이 아니라, 다른 곳에서도 다른 의견을 듣기는 어렵다.
그런데 기후와 사회 프로젝트\footnote{기후와 사회 프로젝트(Climate and Community Project, CCP)는 기후 변화와 불평등의 해소를 목표로 하는 미국의 싱크탱크이다.}와 캘리포니아 주립대 데이비스 캠퍼스가 공동 수행한 한 연구에서 바로 그 ``다른 의견''을 제시한다.
% It is an important piece of scholarship, arguing that we may not need to accept a future in which the mining industry — with the blessing of governments — continues to tear up the world’s forests and occupy its deserts.
이 연구 결과는 채광 기업이 정부를 등에 업고 온 세상의 숲을 뒤집어 엎는 것을 두고만 보지 않아도 된다고 시사하기에 특히나 더 귀중하다.
% Focusing on lithium consumption, the report models different developmental pathways for the U.S., using variables like car ownership, the size of E.V. batteries, city density, public transit, and battery recycling.
리튬 소비량을 예로 들어 보자.
이 연구 결과에 따르면 \triangle차량 소유주 수, \triangle전기차 배터리 용량, \triangle도시 인구 밀집도, \triangle대중교통, \triangle배터리 재활용 수준에 따라 미국의 미래 발전 시나리오를 네 가지로 나눌 수 있다.
% The worst-case scenario, the authors claim, would result in major lithium extraction, as PDAC attendees expect.
연구 저자들은 최악의 시나리오가 곧 PDAC 참가자들이 원하는 미래라고 말한다.
% But they show that reducing the size of E.V. batteries could shrink expected U.S. lithium demand in 2050 by 42 percent.
동시에 전기차 배터리 용량을 줄이면 2050년까지 미국의 리튬 수요를 42퍼센트까지 줄일 수 있다고 분석했다.
% (Car companies, it’s worth noting, are not trending toward smaller vehicles.
(첨언하건대, 소형차는 요즘 자동차 업계의 트렌드가 아니다.
% In April, G.M. announced that it was ending production of the Chevy Bolt, the company’s smallest E.V., in order to build more electric trucks and SUVs.)
지난 4월, GM은 자사의 소형 전기차인 쉐보레 볼트의 생산을 중단하고, 여력을 전기 트럭이나 SUV 생산으로 돌리겠다고 선언한 바 있다.)
% But even if average battery size were to remain the same, cutting car ownership rates, largely by creating denser cities and better public transit, could shrink total lithium demand by somewhere between 18 percent and 66 percent, according to the study.
설령 전기차 배터리 용량을 줄이지 않더라도 대중교통 시설을 정비하고 도시 인구를 집중시켜 차량 소유주 숫자를 줄이면 리튬 수요를 18퍼센트에서 66퍼센트까지도 줄일 수 있다고 한다.
% 

% The spread, or not, of recycling and reusing minerals is another crucial variable.
광물 재활용과 재사용도 무척이나 큰 영향을 미친다.
% John Thompson, the longtime industry insider I spoke to, attended PDAC this year in part as a representative of Regeneration, a company that plans to re-mine abandoned sites and use the profits to restore their original ecosystems.
존 톰슨 씨는 나와 오래전부터 교류하던 소위 ``업계인''이다.
올해 PDAC에 톰슨 씨는 폐광을 재개발하고 그 수익으로 자연을 복원하는 기업인 ``리제너레이션(Regeneration)''의 대변인 자격으로 참석했다.
% He’s a recycling proponent, and hopes that the rest of the industry will catch on soon.
재활용 운동가인 그는 광산업계가 빨리 재활용을 시작하길 바란다고 했다.
% “Everybody would say recycling is important,” he told me.
% But “most people in this conference aren’t interested.”
``재활용이 중요하지 않다고 하는 사람은 아무도 없어요. 여기 있는 사람들 중에 재활용에 관심이 있는 사람도 많이 없을 뿐이죠.''
톰슨 씨는 그렇게 털어놓았다.
% 

% Even the most optimistic version of the future, involving reduced demand and robust recycling, will still require some mining.
적극적으로 수요를 줄이고 재활용을 해서 이상적인 시나리오를 만들어 내더라도 채광 작업을 완전히 없앨 수는 없다.
% What this ought to look like increasingly preoccupies Patrick Donnelly, who works for the Center for Biological Diversity in Nevada.
미국 네바다 주 생물 다양성 보전 센터에서 일하는 패트릭 도넬리 씨는 이런 현실에 매일 밤 잠을 설친다.
% I know Donnelly — as do a lot of other journalists in the West who cover extractive industries — as a ferociously dedicated conservation advocate.
서양 채광 산업을 다루는 기자라면 누구나 알겠지만, 내가 아는 도넬리 씨는 극렬 자연 보호 활동가다.
% A few years ago, Donnelly realized that no one was tracking all of the American lithium projects and decided to do so himself.
몇 년 전, 도넬리 씨는 미국에서 벌어지는 리튬 채굴 작업을 추적하는 사람이 아무도 없다는 점을 깨닫고 직접 팔을 걷고 나섰다.
% His map now shows more than 115 potential mines, clustered in his home state.
지금까지 그는 광산이 될 가능성이 있는 장소를 네바다 주에서만 115 군데 이상 찾아냈다.
% “It’s the biggest mineral rush of our lifetime,” he told me over the phone.
``일생에 한 번 볼까말까한 규모로 사람들이 광산업으로 몰려들고 있어요.''
나와 나눈 전화통화에서 도넬리 씨는 말했다.
% 

% In an ideal world, Donnelly said, the U.S. government would put a moratorium on speculative claims and instead survey all of the country’s mineral deposits in order to identify the least harmful places to mine.
도넬리 씨가 꿈꾸는 이상적인 결말은 미합중국 정부가 투기에 가까운 채굴 신청에 먼저 제동을 걸고, 광산이 들어서도 가장 피해가 적은 지역을 찾아 미주 전역의 광산 자원을 조사부터 하는 것이다.
% This isn’t happening and won’t anytime soon.
그런 일은 일어나지 않았고, 조만간 일어나지도 않을 것이다.
% In May, the U.S. fast-tracked a manganese and zinc mine in Arizona, the first mining project added to a program designed to expedite the clean-energy transition and other infrastructure developments.
지난 5월, 미국은 애리조나 주에 망가니즈와 아연 채광을 신속처리안건으로 통과시켰다.
이 광산은 기반 시설을 확충하고 청정 에너지로 전환하기 위한 첫 발짝이었다.
% But Donnelly also fears that anti-mining sentiment is turning people against electric cars — and against lithium extraction altogether.
동시에 도넬리 씨는 채광 반대론이 사람들에게 전기차는 나쁘고, 리튬 채굴 자체가 나쁜 일이라는 인식을 심어줄까 걱정이다.
% “There is zero chance we can recycle our way out of the problem,” he said.
``재활용만으로 이 문제를 해결할 방법은 존재하지 않아요.''
% This is true.
도넬리 씨의 지적은 정확하다.
% There isn’t enough lithium on the market for battery recycling to realistically meet present demand, let alone the expected increase.
당장 시장에서 재활용 가능한 배터리를 모조리 긁어모아도 미래에 증가할 수요는 커녕 현재 리튬 수요를 맞추기에도 역부족이다.
% “There is an element of the mining resistance movement that opposes not just particular mines but all lithium and all electric vehicles,” Donnelly went on.
``채광 반대 운동가들 중에는 특정 광산뿐만 아니라 모든 리튬 채굴 작업과 전기차 산업을 거부하는 분들이 있어요.''
도넬리 씨는 말을 이었다.
% “Unless we’re talking about deindustrializing society, which I don’t think appeals to most people, we need to be thinking about how and where we’re getting our lithium, and critically examine our own use of these minerals, like the cell phone I’m speaking to you on now, with minerals from South America, where locals say the mines are destroying their environment and community.”
``우리는 탈산업화를 바라는 게 아니에요. 아마 거의 모든 사람들이 저랑 비슷할 겁니다. 그러면 우리가 어디서 어떻게 리튬을 구할지 곰곰히 생각해봐야 합니다. 지금 저희가 통화하고 있는 이 휴대폰처럼 리튬이 어디에 어떻게 사용되는지도 고찰해야 하죠. 남아메리카 지역민들이 말하듯이, 광산이 환경과 사회를 망치고 있는지도 생각해 봐야 합니다.''
% 

% Such are the paradoxes of the globalized green economy, in which blocking a mine in one place means shifting extraction somewhere else.
이것이 바로 국제 녹색 경제의 역설이다.
광산 한 곳을 봉쇄하면 다른 곳에서 광산이 열리기 마련이다.
% We want to decarbonize, yet our lives require ever-increasing supplies of energy.
탄소 배출을 줄이자고 목소리를 높이면서도 에너지 소비량은 줄어들기는 커녕 더욱 늘어만 간다.
% And so climate-minded consumers and the mining industry are locked in a self-justifying embrace.
길국 기후 변화를 걱정하는 소비자와 광산업계는 빠져나올 수 없는 자기합리화의 굴레에 빠져든다.
% We buy an E.V. and think we are doing right by those vulnerable to rising temperatures and tides.
우리는 전기차를 사면서 평균온도와 해수면 상승의 위협을 제일 먼저 받는 취약계층에게 도움을 준다고 생각한다.
% But in trying to continue consuming as we are used to, buying stuff and zipping down the highway, we have exposed many of those same vulnerable people to another threat — the market’s readiness to kill, poison, and displace them to get minerals and metals.
그러면서 소비 습관은 바꾸지 않는다.
사던 대로 물건을 사고, 달리던 대로 고속도로를 질주하다 보면 우리는 취약계층을 또 다른 위협으로 몰아넣는다.
채광을 위해서라면 노동자가 목숨을 잃든, 독극물에 중독이 되든, 삶의 터전을 잃든 아랑곳 않는 ``시장''이라는 위협에 말이다.
% The mining industry, meanwhile, benefits from the self-satisfied consumerism of the E.V. buyer.
광산업계는 그 사이에서 전기차 구매자의 자기만족로부터 이익을 뽑아낸다.
% For all of its disdain for environmentalists, the industry needs green consumers who seek absolution for their carbon-intensive ways of life.
환경운동가들의 반대와 맞서려면 광산업계는 ``녹색 소비자''들이 탄소 배출로 가득한 삶을 유지할 면죄부가 되어야 한다. 
% With their complacent inattention to the injustices inflicted by the green economy, these consumers not only fund the industry’s expansion but give it moral cover.
소비자들은 자기만족감에 녹색 경제의 부당함으로부터 눈을 돌린다.
이런 소비자들이 바로 광산업의 확장에 기름을 붓고, 윤리적인 방패까지 들려 주는 이들이다.
%

\newpage

\title{Translation Journal}
\maketitle
중간 번역 과제로 닉 볼린이 쓴 잡지 기사 ``Good Prospect | Mining Climate Anxiety for Profit''을 한국어로 번역했다.
이 글에서 저자는 채광 산업으로 피해를 입은 사람들의 이야기를 조명하고, 광산업계 회사들의 무책임함과 소비자들의 무관심, 그리고 ``녹색 경제''의 아이러니함을 꼬집는다.
특히, 캐나다에서 진행된 PDAC 행사에 직접 참여하여 취재한 내용을 바탕으로 아메리카 대륙의 희토류 채굴 작업을 집중해서 다룬다.

\section{원문 분석}

먼저 글의 전체적인 어투를 정하기 전에 원문이 어디에 기고된 글인지, 글쓴이는 누구인지 조사해야 한다.
한국어에는 매체마다 자주 쓰이는 문체가 있으므로 출처를 알지 못하면 원문이 드러내고자 하는 느낌을 잘 살릴 수 없을 뿐더러 어색한 번역이 나올 수 있다.
예를 들어, 같은 영어 문장이라도 이 문장이 뉴스 앵커가 한 말이라면 ``해요체''로 번역해야 하고, 신문 기사에서 발췌한 문장이라면 ``하다체''로 번역하는 것이 자연스럽다.

이 텍스트는 \textit{The Drift}라는 문학 전문 잡지에 실린 글이다.
문학 잡지를 표방하고 있는 만큼 지면 대부분이 소설이나 시집에 관한 기고문으로 이루어져 있다.
하지만 시기에 따라 몇몇 시사 기고문도 함께 출판하는 듯 하다.
한국어 잡지에서는 약간의 구어체가 묻어나는 ``하다체''를 자주 볼 수 있으므로 번역문 역시 ``하다체''를 기본으로 하기로 결정했다.

저자 닉 볼린(Nick Bowlin)은 미국의 독립언론사 \textit{High Country News}의 편집자이다.
프리랜서 언론인으로 일하면서 \textit{The Guardian}, \textit{ProPublica}와 같은 큰 신문에 칼럼을 기고한 적이 있다고 한다.
작업물을 훑어보면 그의 전문 분야는 노동조합 활동과 에너지 산업 전반에 관한 내용으로 보인다.
원문 역시 재생 가능 에너지 산업과 착취당하는 취약 계층에 관한 이야기를 다루고 있다.

글 형식은 르포르타주처럼 구성이 되어 있다.
이번 번역 범위에는 포함되지 않았지만, 기사 초반에에 볼린은 광맥이 형성되는 과정을 마치 이야기를 들려주듯이 설명하고, 그 다음 PDAC 행사에서 있었던 일들을 상세하게 그린다.
초벌 번역 전에 글을 읽으면서 잡지 기사보다 짧은 소설을 읽는 것 같다고 생각했다.

\section{번역할 때 고려한 점}

가장 먼저 고려한 점은 문체이다.
첫 글을 읽었을 때 여느 잡지 기사보다 문학적 표현이 많다고 느낀 만큼, 번역문에서도 원문의 연극적인 느낌을 살릴 수 있도록 고민했다.
특히 번역 범위 첫 문단에서 볼린은 광산업이 주민들에게 끼친 피해를 나열하는데, 한국어 번역본에서도 영어에서와 같이 그냥 피해 사실을 담담히 나열하자니 영어 원문의 느낌이 잘 살지 않았다.
이 부분은 각 사례를 쪼개서 질문 형식으로 나열하는 것으로 해결했다.
또, 영어에서 인터뷰 내용을 인용한 뒤에 ``he/she said'' 따위의 문장을 덧붙이는 일이 많은데, 이 부분은 한국어 신문 기사보다는 한국어 소설에서 쓰는 방식을 모방했다.

다음으로 고려한 부분은 고유명사의 처리이다.
이 글에서는 사람 이름은 물론이고, 회사 이름이나 ``퍼스트 네이션''처럼 수많은 고유명사가 등장한다.
\href{http://loanword.cs.pusan.ac.kr/}{부산대학교 인공지능연구실}에서 외래어 자동 변환기를 제공해주기는 하지만, 실제로 발음을 분석해서 외래어 표기법에 맞게 한국어로 음차하는 것이 아니라 자주 등장하는 인명, 지명 등을 모은 데이터베이스에서 단어를 검색하는 방식이므로 필요한 단어를 찾을 수 없을 때가 많았다.
멀리 갈 것도 없이 이 글의 저자인 닉 볼린의 성 Bowlin도 찾을 수 없어서 볼링(Bowling)으로부터 유추해서 옮겨야 했다!
아메리카 원주민 성씨는 더욱 찾기 힘들어서 같은 성씨를 쓰는 유명인을 먼저 찾고, 그 사람이 언급된 뉴스를 직접 듣고 한글로 옮겼다.
본문에 등장한 릭 치추(Rick Cheechoo)는 캐나다 아이스하키 선수인 조나단 치추(Jonathan Cheechoo)를 먼저 찾고, 조나단 치추가 출전한 경기 해설진의 발음을 옮긴 것이다.

``퍼스트 네이션''과 같이 한국어에 없는 개념을 어떻게 글에 녹여낼지도 고민했다.
첫 번역본에서는 퍼스트 네이션을 아메리카 대륙에 원래부터 살고 있던 사람들이라는 의미에서 ``원주민''이라고 번역했는데, 이렇게 번역하면 퍼스트 네이션 사람들의 정체성을 무시하는 일이 될 수 있다는 지적을 듣고 ``퍼스트 네이션''이라고 그대로 음차하되, 각주를 달아 퍼스트 네이션이 정확히 무엇을 의미하는지 설명했다.
그 외에도, 한국어 화자가 잘 모를 수 있는 개념도 아래에 각주를 달았다.
첫 문단의 ``블러드 다이아몬드''는 아프리카 사람들에게 주는 피해를 더 잘 나타내기 위해 일부러 ``블러드 다이아몬드''라고 음차하지 않았다.

제목을 번역할 때는 ``Good Prospect''가 시사하는 긍정적인 면모와 본문이 서술하는 부정적인 면모 사이의 괴리감을 나타내고자 하였다.
번역본에서는 ``탐광하기 좋은 날''이라고 적었는데, 1. 이 글이 기고된 잡지가 원래는 문학 잡지라는 점, 2. 한국에서 교육을 받은 사람이라면 누구나 연상할 수 있는 단편 소설이라는 점에서 현진건의 ``운수 좋은 날''을 오마주했다.

또, 부제목인 ``Mining Climate Anxiety for Profit''에서 활용한 ``doing ~ for profit'' 형태는 인터넷에서도 일종의 밈(meme)이나 숙어로도 자주 사용한다.
다만 한국어는 누구나 알아들을 수 있을 만큼 밈이 퍼져 있지 않으므로 ``땅 파면 돈이 나오나''라는 숙어를 변형했다.

\section{더 개선할 점}

시간이 더 주어진다면 한국어 잡지를 더 읽고 잡지 문체에 좀 더 익숙해진 다음에 한 번 더 퇴고하는 기회를 마련하고 싶다.
과제를 하는 기간 동안 읽을 수 있었던 잡지는 미용실에서 차례를 기다리는 동안 읽은 여성지가 전부였기 때문에 문체에 대한 사전조사가 부족했다고 느꼈다.
특히 시사 현안이나 연구 결과를 인용하는 부분을 번역할 때 자연스럽게 신문 기사를 번역하듯이 어투가 바뀌었다.
이 부분은 시사IN이나 경제지같이 현안에 민감한 주간지를 더 읽어보면 고칠 수 있을 것 같다.

또, 일상적으로 사용하는 비문이 또 없는지 더 검증할 필요가 있겠다.
``~로서''와 ``~로써''의 차이, ``~던''과 ``~든''의 차이, ``예시''와 ``예''의 차이처럼 눈에 잘 보이지 않는 차이로 문장의 뜻이 완전히 달라지는데, 이런 차이를 눈치채려면 맞춤법 검사기를 사용하거나 사람이 직접 검증하는 수밖에 없다.
아쉽게도 무료로 사용할 수 있는 한국어 맞춤법 검사기 중에서 간단한 오타 외에 문맥에 맞지 않는 단어도 찾아 주는 고급 검사기가 없고, ChatGPT 같은 언어 모델도 한국어 데이터가 적어서 번역투가 묻어 나오는 문장밖에 만들지 못한다.
시간이 더 있다면 며칠 뒤에 다시 처음부터 읽어서 비문이 있는지 다시 한번 확인하고 싶다.
\end{document}
