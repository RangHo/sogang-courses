\documentclass{translation}

\firstname{Juhun}
\lastname{Lee}
\instructor{Professor Hyejin Kim}
\course{Translation Practicum}
\date{17 April 2024}

% A Good Prospect | Mining Climate Anxiety for Profit
\title{탐광하기 좋은 날 | 기후 불안을 파서 금을 캐는 방법}

\begin{document}

% The mining industry has always provided an economic justification for the displacement and exploitation of people all over the world: the colonization of the Americas for gold, silver, iron, and copper; blood diamonds in West and Central Africa; child labor in cobalt mines in the DRC; and thousands of deaths linked to paramilitaries financed by multinational mining companies in Colombia.
채광 업계에서는 언제나 경제 발전을 명분 삼아 주민들을 강제로 이주시키고 착취해왔다.
금, 은, 철, 구리를 이유로 아메리카 대륙을 식민지로 삼은 이들이 있지 않은가?
서아프리카와 중앙아프리카에서 수입하는 피투성이 다이아몬드\footnote{블러드 다이아몬드(blood diamond)는 아프리카 분쟁지역에서 전쟁 자금 조달을 위해 생산되는 다이아몬드를 말한다.}는 또 어떤가?
콩고의 코발트 광산에서 곡괭이를 드는 이는 아이들이 아닌가?
콜롬비아에서는 거대한 채광 기업이 후원하는 준군사조직이 수천 명을 죽이지 않았는가?
% The move to renewable energy will likely expose the poorest people on the planet to more of the same from these fierce extractive forces.
이런 일을 벌인 이들이 이제는 재생 가능 에너지 전환을 명목으로 세계 최빈곤층을 착취할 것이다.
% A Nature study published late last year found that more than half of the materials needed for the green energy transformation are located on or near relatively undeveloped land where Indigenous and peasant populations live.
작년 『네이처』 지에 수록된 한 연구는 녹색 에너지 전환에 필요한 원자재의 반 이상이 아메리카 원주민과 소농민이 모여 사는 저개발 지역에 집중되어 있다는 결과를 내놓았다.
% As mines encroach on these communities, they will be removed from their homelands or forced to live with profound industrial pollution.
광산이 이러한 지역을 스멀스멀 잠식하면 이들은 결국 고향을 떠나거나, 환경 오염을 매일같이 견디며 살아갈 수밖에 없다.
%

% The focus at this year’s PDAC was primarily on the economic benefits for these communities.
올해 PDAC 행사에서는 광산업이 원주민들에게 가져다주는 경제효과에 집중했다.
% First Nations speakers attested that mining can be good for Indigenous communities.
퍼스트 네이션\footnote{퍼스트 네이션(First Nation)은 유럽인들이 아메리카 대륙에 도착하기 전 캐나다 지역에 정착한 원주민들이 이룬 국가 내지는 공동체이다.} 출신 연사들은 광산업이 원주민에게 좋은 소식이라고 역설했다.
% “Before the mine, we had nothing,” Donny McCallum told the crowd at a panel on Indigenous economic inclusion.
% McCallum is a member of the Marcel Colomb First Nation (MCFN), in Manitoba.
캐나다 마니토바 주에 있는 마르셀 콜롬(Marcel Colomb) 퍼스트 네이션 소속인 도니 맥캘럼 씨는 원주민과의 경제 협력을 다루는 공개 토론회에 참여했다.
곧 토론회 청중에게 그는 이야기를 시작했다.
``광산이 있기 전, 우리에게는 아무 것도 없었습니다.''
% In 2022, the nation signed joint ventures with two contracting companies designed to help secure employment for members at a nearby gold mine.
2022년, 마르셀 콜롬 퍼스트 네이션은 주민들이 근처 금광에서 일할 수 있도록 두 위탁 회사와 공동 투자 계약을 맺었다.
% “We want a piece of the pie,” McCallum explained.
``우리도 파이 한 조각은 가져가야죠.''
% The panel’s moderator, Christian Sinclair, who is Opaskwayak Cree, encouraged Canada’s First Nations to follow MCFN’s example and form Indigenous economic development corporations.
맥컬럼 씨의 설명이 끝나자, 오파스콰약 크리(Opaskwayak Cree) 퍼스트 네이션 출신 사회자 크리스티안 싱클레어 씨는 다른 퍼스트 네이션도 마르셀 콜롬 퍼스트 네이션을 본보기 삼아야 한다고 극찬했다.
% He pointed to the Southern Ute Tribe of southwestern Colorado, which sits atop a rich formation of coal-bed methane.
그러면서 미국 콜로라도 주 남서부에 있는 남부 우테족의 성과도 함께 언급했다.
% The tribe used these resources to build a three-billion-dollar organization.
남부 우테족은 부족 영토에서 뽑아낸 석탄층 메탄가스로 시장 가치 30억 달러짜리 기업을 일구어낸 경력이 있다.
%

% The only note of hesitation about tying the well-being of First Nations to mining industry profits was sounded by an older man with a long ponytail during the Q&A portion of this session.
광산업의 발전은 곧 퍼스트 네이션의 발전이라는 주장은 모두가 만장일치라 받아들이는 듯 했다.
단 한 명 빼고는 말이다.
% “The land cannot sustain making the most amount of money in the least amount of time,” he told the panelists.
``이렇게나 짧은 기간에 큰돈을 만드는 이 사업을 대지가 과연 언제까지 감당할 수 있겠습니까?''
질의응답 시간이 시작되자, 긴 머리를 묶어올린 한 나이 지긋한 남자가 토론회 패널을 향해 따져 물었다.
%

% I caught up with this man, Rick Cheechoo, later on, at a reception for PDAC’s Indigenous Program over mini bison potpies and wild rice salad.
나는 이 남자를 PDAC 토착민 지원 프로그램 피로연장에서 다시 만났다.
% He is a member of the Moose Cree First Nation.
% A large gold mine operates near his nation’s traditional territory, he told me.
무스 크리(Moose Cree) 퍼스트 네이션 출신 릭 치추라고 자신을 소개한 그는 들소 포트파이\footnote{고기와 야채를 넣어 끓인 수프를 그릇에 담아 파이 시트를 덮어 구워낸 파이의 한 종류.}와 줄풀 샐러드를 담으며 나에게 무스 크리 자치령 근처 커다란 금광 이야기를 해 주었다.
% There have been benefits — he described agreements between the mining company and the First Nation that provide jobs and healthcare — but a company’s drive for profit puts it at odds with some tribal needs, especially preserving their culture and treaty rights and ensuring that families displaced by industry are compensated.
취업이 보장되고, 건강 보험도 회사 측에서 들어주는 등 이득이 없는 것은 아니었지만, 기업은 이익을 좇아 종종 원주민들과 충돌했다.
원주민의 문화와 자주성를 존중하려는 노력도, 집을 떠나야 했던 가족에게 지급하는 보상금도 부족했다.
%

% Historically, mining has brought disruption and violence to Canada’s First Nations, and there’s no reason to believe that’s changing.
광산업은 캐나다의 여러 퍼스트 네이션에게 폭력을 행사하며 원주민들을 불안의 구렁텅이에 빠뜨려 왔다.
여태까지 그러하였듯, 앞으로도 변함없이 그러할 터다.
% The past decade has seen numerous confrontations, with protestors blockading mines and arrested en masse by militarized police.
지난 10년은 특히 충돌이 빈번했다.
시위대는 광산을 막으며 목소리를 높였고, 곧 무장 경찰이 대거 연행해 가는 일이 반복됐다.
% Even as the conference was happening, members of the Naskapi and Innu nations were fighting an iron mine in Quebec — a situation that, as far as I could tell, was not addressed by anyone at PDAC.
PDAC 컨퍼런스가 한창인 와중에도 나스카피(Naskapi) 퍼스트 네이션과 이누(Innu) 퍼스트 네이션은 캐나다 퀘벡 주에서 철광과 맞서 싸우고 있었다.
내가 지금까지 만난 그 누구도 이들의 목소리를 대변해주지 않았다.
\end{document}
