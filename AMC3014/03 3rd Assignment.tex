\documentclass{translation}

\firstname{Juhun}
\lastname{Lee}
\instructor{Professor Hyejin Kim}
\course{Translation Practicum}
\date{27 March 2024}


% 8 Tips to Write Better Essays in English
\title{영어 글을 더 잘 쓰는 8가지 방법}

\begin{document}
% Your teacher hands you a graded essay
선생님께서 첨삭이 끝난 작문 과제를 건네주십니다.
% What do you look at first?
제일 먼저 무엇이 눈에 들어오나요?
% Most college students turn their attention to the letter grade or percentage score.
보통 대학생들은 몇 점을 받았는지부터 확인하곤 합니다.
% If it’s high, they are happy.
% If it’s low, they are disappointed.
점수가 높으면 높은 대로 기분이 좋아지고, 낮으면 낮은 만큼 실망스럽지요.
% Many students end the review process at this point.
학생 대부분은 이 정도로 검토가 끝났다고 생각합니다.

% What about you?
여러분은 어떠신가요?
% If you want to write better essays, you will need to understand the criteria teachers use to score them.
작문 실력을 키우려면 선생님의 채점 기준을 잘 이해해야 합니다.

% Develop your thesis
\section{분명한 논제}

% A thesis is the essence of your paper—the claim you are making, the point you are trying to prove.
논제(\textit{thesis})는 글쓰기의 정수입니다.
논제는 곧 주장하고자 하는 의견이고, 증명하고자 하는 사실이거든요.
% All the other paragraphs in your essay will revolve around this one central idea.
글의 모든 문단은 이 논제를 중심으로 연관이 있거나 뒷받침하는 내용이어야 합니다.
% Your thesis statement consists of the one or two sentences of your introduction that explain what your position on the topic at hand is.
주제문(\textit{thesis statement})은 한두 문장 정도로 주제가 무엇인지 소개하고 이에 대한 글쓴이의 의견을 밝히는 문장입니다.
% Teachers will evaluate all your other paragraphs on how well they relate to this statement.
채점할 때 선생님은 글의 다른 문단이 이 주제문을 얼마나 잘 지탱하는지 확인합니다.
% To excel in this area, ask yourself these questions:
여기서 점수를 잘 받으려면 아래 세 가지 질문에 답하며 글을 써 보세요.

\begin{itemize}
  % - Have I clearly introduced my thesis in the introductory paragraphs?
\item 주장하고자 하는 바를 글의 도입부에 분명하게 드러냈는가?
  % - Does the body of my essay support my thesis statement?
\item 본문의 내용이 주제문을 뒷받침하는가?
  % - Does my conclusion show how I have proven my thesis?
\item 결말부에서 논제를 잘 증명했는지 나타내는가?
\end{itemize}

% Strong form
\section{탄탄한 구조}

% A good essay presents thoughts in a logical order.
잘 쓴 글은 글쓴이의 생각을 논리적으로 전개합니다.
% The format should be easy to follow.
따라서 글의 구조도 이해하기 쉬워야 하지요.
% The introduction should flow naturally to the body paragraphs, and the conclusion should tie everything together.
도입부에서 본문으로 넘어갈 때는 자연스럽게 읽혀야 하고, 결말부는 이전까지 이야기한 내용을 한 데 묶어 정리해야 합니다.
% The best way to do this is to lay out the outline of your paper before you begin.
글의 구조를 잘 잡으려면 개요짜기가 제일 좋은 방법입니다.
% After you finish your essay, review the form to see if thoughts progress naturally.
글을 다 쓰고 나서 사고 과정이 자연스럽게 드러나도록 짜임새가 잘 갖춰졌는지 다시 확인해 보세요.
% You might ask yourself:
아래 네 가지를 자문하며 글을 쓰면 더 좋습니다.

\begin{itemize}
  % Are the paragraphs in a logical order?
\item 문단 순서가 논리적인가?
  % Are the sentences of each paragraph organized well?
\item 문단 안의 문장은 논리적으로 구성되어 있는가?
  % Have I grouped similar pieces of information in the same paragraph?
\item 연관된 정보를 한 문단에 제대로 모았는가?
  % Have I included transitions to show how paragraphs connect?
\item 문단 사이의 관계가 잘 나타나도록 올바른 이음말을 사용했는가?
\end{itemize}

% Style
\section{깔끔한 문체}

% Just as your clothes express your personality, the style of your essay reveals your writing persona.
옷이 사람의 성격을 드러내듯, 문체는 글의 성격을 드러냅니다.
% You demonstrate your fluency by writing precise sentences that vary in form.
필요에 따라 정확하게 문장의 형식을 바꾸면서 작문 실력을 보여줄 수 있지요.
% To illustrate, a child might write robotically: I like to run. I like to play. I like to read, etc.
예시를 들어 볼까요?
아이들이 쓰는 문장은 꽤 천편일률적입니다. ``저는 달리는 게 좋아요'', ``저는 노는 게 좋아요'', ``저는 책 읽는 게 좋아요'', 이렇게요.
% A mature writer uses various types of sentences, idiomatic phrases, and demonstrates knowledge of genre-specific vocabulary.
하지만 노련한 작가는 장르에 따라 단어 선택을 달리하면서 다채로운 문장 구조에 속담이나 고사성어까지 섞어 가며 글을 쓰지요.
% To improve your style, ask yourself:
아래 질문에 답하며 글을 쓰면 문체를 더 갈고닦을 수 있습니다.

\begin{itemize}
  % Will my sentences create an impact on the reader?
\item 읽는이의 기억에 남는 문장인가?
  % Have I used various types of sentences (complex and compound)?
\item 간단한 문장과 복합 문장을 적절히 섞어서 사용했는가?
  % Have I correctly used topic-specific vocabulary?
\item 주제와 관련 있는 단어의 용례가 올바른가?
  % Does the writing sound like me?
\item 나만의 문체를 사용했는가?
\end{itemize}

% Conventions
\section{올바른 맞춤법}

% Conventions include spelling, punctuation, sentence structure, and grammar.
맞춤법은 철자, 문장 부호, 문장 구조, 문법을 모두 포함하는 개념입니다.
% Having lots of mistakes suggests carelessness and diminishes the credibility of your arguments.
맞춤법 오류가 많으면 많을수록 글을 대충 썼다는 인상을 주고, 곧 글의 신빙성이 떨어지지요.
% If you make too many errors, your writing will be difficult to understand.
맞춤법을 너무 많이 틀리면 가독성이 떨어지기도 합니다.
% Wouldn’t it be a shame for a teacher to miss the excellent points you made because of poor grammar?
고작 문법 오류 때문에 선생님이 여러분의 참신한 의견을 보고 지나친다니, 아깝잖아요?
% To avoid this, always use proofreading software, such as Grammarly, to weed out the major errors.
그러니 항상 Grammarly 같은 맟춤법 검사기를 써서 눈에 보이는 문제를 잡아내고,
% Follow up with a close reading of your entire paper.
그 다음 글 전부를 꼼꼼히 읽으며 퇴고해야 합니다.

% Supporting documents and references
\section{정확한 참고 자료}

% Finally, your teacher will examine your resources.
마지막으로 확인하시는 것은 참고 자료 목록입니다.
% Select information from reliable websites, articles, and books.
참고할 정보는 꼭 신뢰할 수 있는 웹사이트나 논문, 책에서 골라 주세요.
% Use quotes and paraphrases to support your ideas, but be sure to credit your sources correctly.
이런 외부 자료를 그대로 인용하거나 문장에 맞게 바꾸어서 논지를 뒷받침하되, 정확하게 출처를 표기해야 합니다.
% Here are some questions to consider:
참고 자료를 표기할 때 아래 세 가지는 꼭 생각해 보세요.

\begin{itemize}
  % Have I demonstrated proof of extensive research?
\item 사전 조사를 성실히 했음을 잘 나타내었는가?
  % Are my main points supported by references, quotes, and paraphrases?
\item 참고 자료나 인용문이 주장하는 바를 뒷받침하는가?
  % Have I used the proper format for my citations—MLA, APA, Chicago?
\item MLA, APA, Chicago 등, 요구하는 글 양식에 맞게 자료 출처를 표기하였는가?
\end{itemize}

\hfill

% Do you want to develop your essay-writing skills?
작문 실력을 늘리고 싶다면 먼저
% Pay attention to the same things your teacher will evaluate.
선생님이 중요하다고 생각하는 부분에 집중해야 합니다.
% The grades you get on your essays are important, but you can never improve your writing if they are the only things you consider.
물론 과제 점수도 중요하지만, 숫자에만 연연하면 절대로 글쓰기 실력을 갈고닦을 수 없지요.
% Focus on improving the overall structure of your essays—the thesis development, form, style, conventions, and support.
주제문과, 구조, 문체, 맞춤법, 참고 자료 등, 글의 전체적인 짜임새에 주목하면서 글을 써야 합니다.
% Learning to master these five elements will cause your scores to soar!
이 다섯 가지만 완벽하게 익혀도 점수가 쭉쭉 올라갈 거예요!
\end{document}
