\documentclass{translation}

\usepackage{hyperref}

\firstname{Juhun}
\lastname{Lee}
\instructor{Professor Hyejin Kim}
\course{Translation Practicum}
\date{22 May 2024}

% In translation: The language of grief 
% My experience thinking in English but feeling in Korean  
\title{비탄의 언어, 한국어}
 
\begin{document}
% My first language is Korean.
나의 모국어는 한국어다.
% I speak it at home, and I know how to read and write in it.
집에 돌아오면 한국어로 이야기하고, 한국어를 쓸 줄도, 읽을 줄도 안다.
% But when watching the news in Korea with my relatives, about 40 per cent of it flies over my head, and I’m left looking at pictures and surreptitiously googling translations away from the curious eyes of my fully fluent cousins.
그런데 한국에서 친척들과 뉴스를 보고 있자면, 아나운서가 내뱉는 말의 40 퍼센트는 한 귀로 들어가 다른 쪽 귀로 흘러나온다.
결국 멍하니 앉아 화면만 보다가, 신기하다는 듯 쳐다보는 ``원어민'' 사촌들 눈을 피해 몰래 번역본을 검색하는 일이 일상다반사다.
% I guiltily turn on subtitles when I watch shows or movies in Korean.
부끄럽지만 나는 한국  TV 프로그램이나 영화를 볼 때 영어 자막을 켜고 본다.
%  

% As a recovering Korean school dropout, I don’t speak Korean easily, but despite this, I always reach for it first when I want to express frustration, anger, and grief.
나는 한국어 학교를 중간에 자퇴하고 말았기에 한국말이 서툴다.
그럼에도 답답하고, 화 나고, 슬픈 일을 이야기할 때면 한국어가 먼저 튀어나온다.
% I translate in front of my therapist, interpreting in real time for myself.
때문에 상담을 받으러 가면 나는 나와 상담사 사이에 서서 동시통역을 하는 통역사가 된다.
% I find English to be such a flimsy and nebulous language, and communicating so-called ‘negative emotions’ in English feels almost comically flat and one-dimensional.
그렇게 말을 하고 있으면 영어가 얼마나 얄팍하고 애매모호한 언어인지 느껴진다.
영어로 나누는 소위 ``나쁜 감정''에 관한 이야기는 어찌 이렇게 일차원적이고 웃음이 나올 정도로 무미건조한가?
%  

% This is one of the unique vexations of the “third culture kid,” a term credited to American sociologist Ruth Useem, who defined this figure as a child who grows up in a culture different from the ones their parents grew up in.
미국의 사회학자 루스 우심은 부모님과 다른 문화를 보고 자란 아이들을 두고 ``제3문화 아이들''이라고 명명했다.
내가 겪는 이 고충이 바로 제3문화 아이들만이 특별하게 경험하는 어려움이다.
% Telling my friends about grief and anger is tiring and rings hollow because of my constant translation.
친구들에게 나의 고통과 분노를 이야기하면 나는 끊임없이 머릿속에서 번역해야 한다.
그러면 피곤할 뿐만 아니라 공허한 말이 되어버린다.
% Talking to my family members in Korean doesn’t cut it either because my Korean lags and stumbles.
그렇다고 띄엄띄엄 내뱉는 한국어로는 부족하기에 가족에게 나의 고통과 분노를 이야기할 수도 없다.
%  

% It does not help that the more I go back to Korea, the more graves I visit.
한국에 돌아갈 때마다 찾아가는 산소가 늘어나는 것도 전혀 도움이 안 된다.
% As I grow older and many of my extended family members reach their 70s and 80s, I have become more fluent in the language of grieving.
나이가 들면서 가족 중에 칠순, 팔순을 넘기는 분들이 늘어났다.
자연스레 죽음을 슬퍼하는 한국어만이 늘었다.
% There are rules, layers of etiquette, and a ceremonial aspect that I cannot help but honour.
지키지 않고 있을 수는 없는 규칙과 예절과 의식이 있기 때문이다.
% I have never grieved in English.
나는 영어로 비탄함을 토로해본 적이 없다.
% Thus, grief in Korean feels heftier, but there is still the strange frustration of knowing all the right words in Korean but being unable to translate them into English, my dominant language.
그러니 한국어로 슬퍼하는 것이 훨씬 무겁게 느껴진다.
동시에, 한국어로는 잘 아는 단어를 주 사용 언어인 영어로 번역할 수 없다는 데에서 오는 이상한 답답함도 존재한다.
% I can’t reach the words I need to express my grief.
슬픔을 표현하고 싶은데 알맞는 단어가 없다.
% It feels like I am stuck by myself, holding this big emotion that does not want to be held and cannot be articulated.
이럴 때면 마치 놓아주어야 할 감정을 말로 형용하지 못한 채 붙잡고 나 자신 사이에 끼여 꼼짝하지 못 하는 듯한 기분이다.
%  

% Even worse, I speak a terrible Frankenstein dialect of Korean that mixes my mother’s Seoul accent and North Korean figures of speech from her mother and grandmother with my father’s countryside twang.
엎친 데 덮친 격으로, 나의 한국어는 어머니로부터 물려받은 서울말, 할머니와 증조할머니가 사용하시는 북한식 표현, 아버지의 시골 말투를 한 데 섞은 잡탕밥이다.
% This odd mix is topped off with slightly rounded-out vowels characteristic of English speakers and lapses in a formal language that — if my actual accent hadn’t done it already — would make it glaringly obvious that I’m not ‘from here’.
여기에 영어 화자의 고질병인 원순모음화와 높임말 실수를 곁들이면 난 ``여기 사람이 아니''라는 게 너무나도 투명히 보인다.
그 전에 억양으로 이미 다들 눈치챘겠지만 말이다.
% Rather, I’m just a ‘gyopo’: a pejorative for diaspora Koreans that implies we’ve lost touch with our roots.
결국, 한국 사람들에게 나는 고향을 잊어버리고 만 ``교포''일 뿐이다.
% It’s one of the many labels family members give me.
적어도 나의 친척 가족은 나를 그렇게 낙인찍었다.
% 

% The result is a grey area: speaking in Korean about rage and grief is frustratingly slow, as my accent and limited vocabulary hinder me, yet translating into English feels lacking and weak.
이렇게 나는 이도저도 아닌 애매한 상황에 놓인다.
한국어로 풀어내는 슬픔과 분노는 부족한 단어와 억양 탓에 답답할 정도로 느리고, 영어로 번역해서 풀어내면 내용 없이 약한 이야기가 된다.
% I don’t speak Korean well enough to see a therapist in Korean, but on the other hand, my vague circling in English of the true meaning I want to get at drives me a little bit insane.
한국어로 상담을 받으러 가기에는 한국어 실력이 모자라 영어로 내 진짜 의도를 설명하고 있으면 조금은 미쳐 가는 가분이 든다.
% I can’t help but think, “This would be so much faster and so much more accurate in Korean!”
``한국어로 이야기하면 훨씬 더 빠르고 정확하게 이야기할 수 있는데!'' 하고 머릿속으로 소리친다.
% I’m not just sad that my grandfather died — I’m angry that he was alone, and it’s unfair that my mother didn’t get to see her dad before he passed, and it feels hollow and twisted.
나는 그저 할아버지가 돌아가셔서 슬픈 것이 아닌데.
할아버지가 홀로 계셨다는 게 화가 나고, 어머니가 당신 아버지를 마지막으로 볼 수 없었다는 게 불공평하고, 모든 게 다 공허하고 뒤틀린 것처럼 느껴지는 건데.
% In Korean, these feelings are not discrete: they happen all at once.
한국어로는 이 모든 일이 따로 벌어지지 않는다.
모든 일이 한 번에 일어난다.
%  

% For example, how do you begin to describe the expression ‘eoieopda’ (어이없다) in English anyway?
예를 들어 보자.
``어이없다''는 표현을 영어로는 어떻게 설명해야 할까?
% Maybe it’s when you’re off-kilter because the situation at hand is plain dumbfounding but with a tinge of unfairness and disbelief.
약간은 불합리하고 믿기지 않는 상황에 말문이 턱 막히고 머리가 고장나버린 상황이라고 이야기할 수 있겠다.
% You can use it when someone acts rude out of the blue or when your grandfather’s condition declines sharply overnight.
누군가가 아무 이유도 없이 무례하게 군다거나 병원에 계신 할아버지의 용태가 하룻밤 사이에 급격히 악화됐을 때 사용하는 말이라는 게 적절한 예시일 것 같다.
% What about ‘jeong’ (정)?
``정(情)''은 어떨까?
% I think it’s remembering people’s birthdays even if they mistreated you, or texting your friend a picture of something that reminds you of them.
내가 느끼기에 정은 내게 잘못한 사람도 생일은 챙겨 주는 것이고, 친구 생각이 나게 하는 사진을 본인에게 문자로 보내는 것이다.
% It’s attachment but more than that.
애착 관계이기도 하지만 동시에 그 이상이다.
% You can have jeong for people, objects, places, and situations.
사람 뿐만 아니라, 물건에도, 장소나 상황에도 정을 붙일 수 있게 때문이다.
% In the case of grieving, jeong is love with nowhere to go.
누군가의 죽음을 경험했을 때의 정은 갈 곳 잃은 사랑이다.
%  

% Luckily, I know the secret to untangling my emotions that get lost in translation: I’ll grieve as much as I want, in as strange a cocktail of languages as I want, and then I’ll write about my experience over and over until I’m satisfied.
다행히도 언어의 장벽을 넘으면서 잃어버린 감정을 풀어내는 방법을 한 가지 나는 알고 있다.
얼마나 이상하게 섞었든, 나는 나의 ``언어 칵테일''로 비탄에 잠길 것이다.
그리고 내가 만족할 때까지 몇 번이고 이 경험을 글으로 남길 것이다.
% But until I’m done, please don’t say anything if you see me turning on the subtitles for a Korean movie.
그 전까지는 내가 한국 영화를 볼 때 영어 자막을 켜더라도 못 본 체 해 주었으면 한다.
% It’s a work in progress.
아직 노력하는 중이니까.
\end{document}
