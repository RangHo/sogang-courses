\documentclass{translation}

\usepackage{hyperref}

\firstname{Juhun}
\lastname{Lee}
\instructor{Professor Hyejin Kim}
\course{Translation Practicum}
\date{15 May 2024}

% Data Activists Target OpenAI In Challenge To ChatGPT’s ‘Hallucination’ Problem
\title{개인정보 운동가들이 챗GPT의 ``환각 현상''을 문제 삼아 오픈AI를 저격하다}

\begin{document}
% Privacy activists filed a complaint against ChatGPT maker OpenAI on Monday over the company’s failure to correct misinformation its chatbot regularly “hallucinates” about people, a move that will increase pressure on tech firms to address what has become a well known but difficult-to-fix problem as they race to roll out the AI tools to more customers.
사생활 보호 운동가들이 챗GPT(ChatGPT)가 개인에 관한 잘못된 정보를 계속해서 생성한다는 이유로 개발사인 오픈AI(OpenAI)를 관계 당국에 고발했다.
더 많은 사람들이 인공지능 도구를 사용하는 가운데, IT 기업은 부정확한 정보를 생성하는 문제를 해결하라는 압박이 더욱 거세질 전망이다.
% 

% OpenAI's ChatGPT is known to "hallucinate" information in answers.
오픈AI에서 개발한 생성형 인공지능인 챗GPT는 주어진 질문을 존재하지 않거나 틀린 정보로 답하는 이른바 ``환각 현상''을 겪는다고 알려져 있다.
% 

% KEY FACTS
\section{주요 사건}
% 

% Vienna-based nonprofit noyb, short for “none of your business,” filed a data protection complaint with Austria’s data watchdog, accusing OpenAI of violating Europe’s General Data Protection Regulation (GDPR), the strictest privacy and security law in the world.
오스트리아 빈에 본부를 둔 비영리단체 노비(noyb)는 영어 관용구 ``none of your business(네가 알 바 아니다)''의 머릿글자를 딴 만큼, 개인정보권리를 보장하기 위해 활동하는 단체이다.
이 단체는 29일 오픈AI를 유럽 연합 일반 개인정보 보호법(General Data Protection Regulation, GDPR) 위반으로 오스트리아 개인정보 감시 당국에 고발장을 접수했다.
유럽 연합에서 제정, 집행 중인 GDPR은 세계에서 가장 엄격한 개인정보 보호 및 정보보안 관련 법령으로 잘 알려져 있다.
% “Simply making up data about individuals is not an option,” the group said in a statement about the complaint, which was filed on behalf of an unnamed “public figure” and accuses OpenAI of refusing to correct or erase false information and statements it had made up about the individual.
노비는 오픈AI가 잘못된 개인정보를 생성하는 문제를 해결하지 않고, 이미 생성된 정보도 지우려 하지 않는다며 한 익명의 공인을 대신하여 고발장을 접수했다.
이날 노비는 ``특정 인물에 대한 정보를 만들어내버리는 것은 있어서는 안 되는 일''이라고 고발장에 적었다.
% 

% For example, the group said ChatGPT gave “various inaccurate information” when asked about the figure’s birth date and that OpenAI said “there is no way to prevent its systems” from displaying the false information.
한 사례로, 챗GPT에게 피해자의 생일과 같은 정보를 물어보면 ``여러 부정확한 정보''를 생성해냈고, 이에 대해 오픈AI는 ``챗GPT가 틀린 정보를 생성하는 것을 막을 수 있는 방법은 존재하지 않는다''며 답변을 피했다고 노비는 주장한다.
% Instead, the complaint said OpenAI only offered to block or filter results based on prompts like the figure’s name — which would filter all information about them — something noyb said would still leave the incorrect data in OpenAI’s systems, “just not shown to users.”
오픈AI는 그 대신 피해자에 관한 모든 정보가 보이지 않도록 이름 등 개인정보를 묻는 질문에 대해 답변 생성을 막거나 특정 정보를 차단하는 방안을 제시했지만, 노비는 ``이 조치는 사용자가 보지만 못 할 뿐''이며 챗GPT에 저장된 부정확한 정보를 고치지는 않는다며 반발했다.
% 

% It also accused OpenAI of failing to disclose relevant information about the person when requested, including what data had been processed, its sources and who it had been shared with, a legal obligation noyb lawyer Maartje de Graaf said “applies to all companies,” adding that it is “clearly possible” to keep track of information sources when training an AI system.
또한 오픈AI는 당사자 요청에도 어떤 개인정보가 어떤 출처로부터 누구와 공유되었는지와 같은 정보를 제대로 공개하지 않았다는 의혹도 받고 있다.
마르트예 데 흐라프 노비 소속 변호사는 ``정보 공개 의무는 모든 회사에 적용된다''며 인공지능 학습에 필요한 정보 출처를 기록하는것은 ``당연히 가능한 일''이라고 덧붙였다.
% OpenAI, which has previously acknowledged the problem AI hallucinations pose to tools like ChatGPT, did not immediately respond to Forbes’ request for comment.
이에 관해 포브스는 오픈AI에게 의견을 요청했으나 곧바로 답변하지 않았다.
오픈AI는 이미 ``환각 현상''이 챗GPT와 같은 인공지능 도구에 여러 문제를 일으킨다는 점을 시인한 바 있다.
% 

% CRUCIAL QUOTE
\section{중요 인터뷰}
% 

% “Making up false information is quite problematic in itself. But when it comes to false information about individuals, there can be serious consequences,” de Graaf said in a statement.
데 흐라프 변호사는 ``잘못된 정보를 만들어내는 것도 문제지만, 특정 인물에 대한 틀린 정보는 더더욱 큰 파장을 일으킨다''고 설명했다.
% “It’s clear that companies are currently unable to make chatbots like ChatGPT comply with EU law, when processing data about individuals. If a system cannot produce accurate and transparent results, it cannot be used to generate data about individuals. The technology has to follow the legal requirements, not the other way around.”
그는 ``기업들이 현재 챗GPT와 같은 챗봇을 유럽 연합 규제에 따르게 만들 방법이 없다는 점은 명백하다. 만약 챗봇이 정확하고 투명성 있는 정보를 생성할 수 없다면, 그 챗봇은 개인에 관한 정보를 생성할 때 사용되어서는 안 된다. 기술이 법에서 규정하는 기준을 따라야지, 그 반대가 되어서는 안 된다''고 덧붙였다.
% 

% NEWS PEG
\section{뉴스 거리}
% 

% The complaint ups the ante against OpenAI over how it uses and trains the models powering ChatGPT.
이번 고발은 챗GPT 모델의 사용 및 학습 방법에 대한 오픈AI의 책임을 더욱 강화했다.
% The company, in common with many top generative AI makers, is already facing a litany of copyright lawsuits over the data it has used to train its models and a host of other legal issues like the privacy of the data it scraped for training.
오픈AI처럼 생성형 AI를 제작하는 기업들은 이미 수많은 저작권 침해 소송에 직면했고, 모델을 학습하는 데 사용한 데이터에 포함된 개인정보 등 수많은 법적 문제도 해결해야 한다.
% So called hallucinations, where the AI system produces a misleading or false result on a prompt but presents it as if true, are also a growing legal headache, such as a defamation suit from a radio host in Georgia.
질문에 대해 부정확하거나 틀린 정보를 마치 진실인 것처럼 답하는 이른바 ``환각 현상'' 역시 여러 법정 공방의 주제로 떠올랐다.
한 사례로, 미국 조지아 주의 한 라디오 진행자가 오픈AI를 명예훼손으로 고소한 바 있다.
% The complaint is not the first time OpenAI has run up against Europe’s powerful data protection regime and the company was already forced to make changes by Italy’s data protection authority in 2023.
오픈AI는 이미 유럽의 강력한 개인정보 관련 규제에 맟닥뜨린 경험이 있다.
지난 2023년, 이탈리아의 개인정보 보호법이 개정되며 오픈AI도 개발 방침을 바꾸어야 했다.
% 

% WHAT TO WATCH FOR
\section{눈여겨볼 점}
% 

% GDPR is a powerful set of rules that can force a company to make major changes to its operations in order to keep operations going in the world’s largest trading bloc.
GDPR은 세계에서 가장 큰 유통 시장이라는 유럽 연합의 지위를 이용하여 기업들의 운영 방식을 바꿀 수 있는 강력한 법령이다.
% It also empowers regulators to levy fines of up to 4% of global turnover.
동시에 전 세계 매출액의 4\%까지 벌금을 부과할 수 있는 규제 기관의 철퇴이기도 하다.
% Data complaints can evolve to cover the entire EU if the issue stretches beyond one country’s borders, with investigations happening through cooperating watchdogs.
개인정보 관련 수사가 국경을 넘어 다른 유럽 국가까지 연루되어 있다면 EU 회원국의 모든 개인정보 감시 당국이 협력하여 조사하는 거대한 수사로 번질 수도 있다.
% Noyb said it expects the matter will be dealt with in such a manner, potentially upping the stakes for OpenAI, though investigations can take years to resolve.
노비는 이 사안도 같은 규모로 수사하여 오픈AI가 제대로 기준을 충족하게 되었으면 한다고 밝혔지만, 조사가 완료되기까지 수 년이 걸릴 수도 있다.
% AI firms will be watching the outcome of the case keenly.
AI 업계에서는 상황이 어떻게 흘러가는지 주시하고 있다.
% 

% KEY BACKGROUND
\section{주요 배경}
% 

% Noyb has been a potent force within the European data protection space since its founding in 2017, bringing a total of 839 cases resulting in €1.74 billion ($1.86 billion) in fines.
노비는 2017년 창립 이래로 총 839건의 고발장을 접수하고 총 17억 4천만 유로(약 2조 5700억 원) 규모의 벌금을 유도하여 유럽 개인정보 보호를 이끄는 선봉대 역할을 해 왔다.
% Its cofounder, Max Schrems, is the activist and lawyer behind some of the most devastating challenges to data sharing deals between the U.S. and EU for major companies like Meta.
막스 슈렘스 노비 공동 창립자는 개인정보 보호 운동가이자 메타(Meta)와 같은 대기업이 미국과 유럽 사이에 개인정보를 공유할 수 있도록 맺은 협약에 제동을 건 변호사이다.
% Schrems’ challenges ultimately overturned two of these major deals — the Privacy Shield and the EU-U.S. Safe Harbor — which forced companies to rethink online business.
슈렘스의 활동은 미국과 유럽 연합 간의 주요 협약 중 ``개인정보 방패 협약''과 ``안전 항만 협약''을 뒤집어 기업들로 하여금 온라인 사업을 재고하게 만들기도 했다. 
\end{document}

