\documentclass{translation}

\firstname{Juhun}
\lastname{Lee}
\instructor{Professor Hyejin Kim}
\course{Translation Practicum}
\date{10 April 2024}

% A Good Prospect | Mining Climate Anxiety for Profit 
\title{탐광하기 좋은 날 | 기후불안을 파서 금을 캐는 방법}

\begin{document}

% The mining industry has always provided an economic justification for the displacement and exploitation of people all over the world: the colonization of the Americas for gold, silver, iron, and copper; blood diamonds in West and Central Africa; child labor in cobalt mines in the DRC; and thousands of deaths linked to paramilitaries financed by multinational mining companies in Colombia.
채광 산업은 언제 어디서든 주민들을 강제 이주하고 착취할 경제적인 빌미를 제공해 왔다.
금, 은, 철, 구리를 빌미로 아메리카 대륙을 식민지로 삼은 이들이 있지 않은가?
서아프리카와 중앙아프리카에서 수입하는 피투성이 다이아몬드는 또 어떤가?
콩고의 코발트 광산에서 곡괭이를 드는 이는 아이들이 아닌가?
콜롬비아에서는 거대한 채광 기업이 후원하는 준군사조직이 수천 명을 죽이지 않았는가?
% The move to renewable energy will likely expose the poorest people on the planet to more of the same from these fierce extractive forces.
재생 가능 에너지는 더 많은 사람들을 사업가들의 횡포에 노출할 가능성이 높다.
% A Nature study published late last year found that more than half of the materials needed for the green energy transformation are located on or near relatively undeveloped land where Indigenous and peasant populations live.
작년 네이처 지에 수록된 한 연구는 녹색 에너지 전환에 필요한 원자재의 반 이상이 토착민들이 모여 사는 비교적 덜 발달된 지역에 묻혀 있다고 분석했다.
% As mines encroach on these communities, they will be removed from their homelands or forced to live with profound industrial pollution.
광산이 이러한 지역을 스멀스멀 잠식하면 원주민들은 결국 고향을 떠나거나, 환경 오염을 매일같이 견디며 살아갈 수밖에 없다.
% 

% The focus at this year’s PDAC was primarily on the economic benefits for these communities.
올해의 PDAC 행사는 원주민들에게 주는 이점에 대한 내용이 주를 이뤘다.
% First Nations speakers attested that mining can be good for Indigenous communities.
캐나다 원주민 연사들은 채광 산업이 토착민들에게는 좋은 소식이라고 역설했다.
% “Before the mine, we had nothing,” Donny McCallum told the crowd at a panel on Indigenous economic inclusion.
% McCallum is a member of the Marcel Colomb First Nation (MCFN), in Manitoba.
캐나다 마니토바 주에 있는 마르셀 콜롬 퍼스트 네이션(MCFN) 소속인 도니 맥콜롬 씨는 토착민과 경제적 협력을 다루는 패널에 출연했다.
``광산이 있기 전, 우리에게는 아무 것도 없었습니다.''
% In 2022, the nation signed joint ventures with two contracting companies designed to help secure employment for members at a nearby gold mine.
2022년, MCFN은 공동체에 소속된 사람들이 근처 금광에서 일할 수 있도록 보장하는 계약을 두 회사와 맺었다.
% “We want a piece of the pie,” McCallum explained.
맥콜롬 씨는 말했다.
``우리도 파이 한 조각은 가져가야죠.''
% The panel’s moderator, Christian Sinclair, who is Opaskwayak Cree, encouraged Canada’s First Nations to follow MCFN’s example and form Indigenous economic development corporations.
패널 사회자이자 오파스콰약 크리족 일원인 크리스티안 싱클레어 씨는 캐나다 원주민들이 MCFN을 본보기 삼아야 한다고 극찬했다.
% He pointed to the Southern Ute Tribe of southwestern Colorado, which sits atop a rich formation of coal-bed methane.
그러면서 미국 콜로라도 주 남서부에 있는 남부 우테족 예시도 함께 들었다.
% The tribe used these resources to build a three-billion-dollar organization.
이 부족은 자신의 땅에 묻힌 석탄과 메탄으로 시장 가치 30억 달러짜리 기업을 만들었다.
% 

% The only note of hesitation about tying the well-being of First Nations to mining industry profits was sounded by an older man with a long ponytail during the Q&A portion of this session.
회의론은 질의응답 시간 때까지 한 마디도 들어볼 수 없었다.
% “The land cannot sustain making the most amount of money in the least amount of time,” he told the panelists.
질의응답이 시작되자, 긴 머리를 묶어올린 한 나이 지긋한 남자가 물었다.
``단기간에 큰돈을 만드는 이 사업을 대지가 감당할 수 있겠습니까?''
% 

% I caught up with this man, Rick Cheechoo, later on, at a reception for PDAC’s Indigenous Program over mini bison potpies and wild rice salad.
나는 이 남자를 PDAC 토착민 지원 프로그램 피로연장에서 다시 만났다.
% He is a member of the Moose Cree First Nation.
무스 크리족 출신 릭 치추라고 자신을 소개한 그는 물소 고기 파이와 줄풀 열매 샐러드를 담고 있었다.
% A large gold mine operates near his nation’s traditional territory, he told me.
우리는 치추 씨의 부족 근처에서 채굴 중인 커다란 금광에 대한 이야기를 나눴다.
% There have been benefits — he described agreements between the mining company and the First Nation that provide jobs and healthcare — but a company’s drive for profit puts it at odds with some tribal needs, especially preserving their culture and treaty rights and ensuring that families displaced by industry are compensated.
물론 직업이 보장되고, 건강 보험도 회사 측에서 들어주는 등 이득 본 일은 많았다.
하지만 기업은 이익을 좇아 종종 부족과 충돌했다.
원주민의 문화를 존중하려는 노력도, 집을 떠나야 했던 가족에게 지급하는 보상금도 부족했다.
% 

% Historically, mining has brought disruption and violence to Canada’s First Nations, and there’s no reason to believe that’s changing.
돌이켜 보면 채광 산업은 캐나다 원주민들에게 불안과 폭력으로 몰아넣어 왔다.
이제 생각해 봐도 예전과 딱히 달라진 점은 없는 듯 하다.
% The past decade has seen numerous confrontations, with protestors blockading mines and arrested en masse by militarized police.
지난 10년은 특히 충돌이 빈번했다.
시위대는 광산을 막으며 목소리를 높였고, 곧 무장 경찰이 대거 연행해 가는 일이 반복됐다.
% Even as the conference was happening, members of the Naskapi and Innu nations were fighting an iron mine in Quebec — a situation that, as far as I could tell, was not addressed by anyone at PDAC.
PDAC 컨퍼런스가 한장인 와중에도 나스카피족과 이누족은 캐나다 퀘벡 주에서 철 광산과 싸우고 있었다.
그 누구도 이들의 목소리를 대변해주지는 않았다.
\end{document}
