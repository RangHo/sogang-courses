\documentclass{translation}

\firstname{Juhun}
\lastname{Lee}
\instructor{Professor Hyejin Kim}
\course{Translation Practicum}
\date{20 March 2024}

% Biden and Trump clinch nominations, setting the stage for a grueling general election rematch
\title{바이든, 트럼프 대선 후보로 확정... 마뜩잖은 재대결 성사돼}

\begin{document}
% WASHINGTON (AP) — President Joe Biden and former President Donald Trump clinched their parties' presidential nominations Tuesday with decisive victories in a slate of low-profile primaries, setting up a general election rematch that many voters do not want.
각 주의 소규모 경선이 잇따라 종료되면서 12일(현지 시각) 민주당과 공화당은 각각 조 바이든 현 미국 대통령과 도널드 트럼프 전 대통령을 후보로 확정했다.
유권자들에게는 달갑지 않은 바이든과 트럼프의 재대결이 성사된 셈이다.

% The outcome of contests across Georgia, Mississippi and Washington state was never in doubt.
조지아와 미시시피, 워싱턴 주의 경선에서는 예상대로
% Neither Biden, a Democrat, nor Trump, a Republican, faced major opposition.
큰 반대 없이 바이든 대통령과 트럼프 전 대통령을 후보로 지명했다.
% But the magnitude of their wins gave each man the delegate majority he needed to claim his party's nomination at the summertime national conventions.
두 후보 모두 과반수의 선거인단을 확보하면서 오는 여름 각 당의 전당대회에서 대선 후보로 지명을 받을 예정이다.

% Not even halfway through the presidential primary calendar, Tuesday marked a crystalizing moment for a nation uneasy with its choices in 2024.
하지만 경선 일정의 반도 채 지나지 않아 2024년 선거에 대한 불만은 더욱 분명하게 드러났다.

% There is no longer any doubt that the fall election will feature a rematch between two flawed and unpopular presidents.
오는 가을 치뤄질 미 대선은 결국 부정 평가로 점철된 두 대통령의 재대결이 확실해졌다.
% At 81, Biden is already the oldest president in U.S. history, while the 77-year-old Trump is facing decades in prison as a defendant in four criminal cases.
바이든 대통령은 올해 81세로 이미 미국 역사상 최고령 대통령 자리를 차지하고 있는 한편, 77세인 트럼프 전 대통령은 4건의 형사 재판과 수십년의 징역형을 선고받을 위험에 처해 있다.
% Their rematch — the first featuring two U.S. presidents since 1912 — will almost certainly deepen the nation’s searing political and cultural divides over the eight-month grind that lies ahead.
전\cdot현직 대통령이 맞붙는 선거는 1912년 이래로 처음이다. 특히 오는 8개월 간 미국의 정치적\cdot문화적 갈등의 골은 더욱 깊어질 것으로 보인다.

% In a statement, Biden celebrated the nomination while casting Trump as a serious threat to democracy.
바이든 대통령은 대선 후보 지명 기념 성명에서 트럼프 전 대통령은 민주주의에의 지대한 위협이라고 주장했다.

% Trump, Biden said, “is running a campaign of resentment, revenge, and retribution that threatens the very idea of America.”
바이든 대통령은 ``트럼프는 혐오와 보복을 선거 운동의 원동력으로 삼는다''며, ``미국이 표방하는 이상을 본질적으로 위협한다''고 단언했다.

% He continued, “I am honored that the broad coalition of voters representing the rich diversity of the Democratic Party across the country have put their faith in me once again to lead our party — and our country — in a moment when the threat Trump poses is greater than ever.”
이어 바이든 대통령은 ``트럼프의 해악과 직면한 지금, 민주당을 이끌기 위해, 더 나아가 미국을 이끌어나가기 위해 나를 뽑아준 수많은 유권자들에게 깊은 감사를 드린다''고 밝혔다.

% Trump, in a video posted on social media, celebrated what he called “a great day of victory.”
반면 트럼프 전 대통령은 사회관계망서비스(SNS)에 이른바 ``위대한 승리의 날''을 축하하는 영상을 게시했다.

% “But now we have to get back to work because we have the worst president in the history of our country,” Trump said of Biden.
트럼프 전 대통령은 바이든 대통령을 두고 ``미국 역사상 최악의 대통령''이라고 비판하며 ``이제 다시 일어나 일할 때''라고 말했다.
% “So, we're not going to take time to celebrate. We'll celebrate in eight months when the election is over."
이어 ``아직은 축하할 때가 아니다''라며, ``우리는 8개월 뒤, 모든 선거가 끝난 뒤 축배를 들 것''이라고 호언했다.

% Both candidates dominated Tuesday's primaries in swing-state Georgia, deep-red Mississippi and Democratic-leaning Washington.
두 후보 모두 경합주인 조지아 주, 공화당 표밭인 미시시피 주와 민주당 지지도가 더 높은 워싱턴 주의 경선에서 압승을 거뒀다.
% Voting was taking place later in Hawaii's Republican caucus.
하와이 주에서는 얼마 뒤 공화당 대회 투표를 진행했다.

% Despite their tough talk, the road ahead will not be easy for either presumptive nominee.
그러나 바이든 대통령과 트럼프 전 대통령의 자신 있는 발언에도 당선까지의 길은 순탄하지만은 않을 전망이다.

% Trump is facing 91 felony counts in four criminal cases involving his handling of classified documents and his attempt to overturn the 2020 election, among other alleged crimes.
트럼프 전 대통령은 4건의 형사 소송에서 기밀문서 유출 및 대선 결과 불복 외, 총 91개 혐의로 재판을 받고 있다.
% He’s also facing increasingly pointed questions about his policy plans and relationships with some of the world's most dangerous dictators.
동시에 독재 정권과의 부적절한 관계와 정책 공약에 대한 반발은 점점 거세지는 실정이다.
% Trump met privately on Friday with Hungarian Prime Minister Viktor Orbán, who has rolled back democracy in his country.
트럼프 전 대통령은 지난 금요일 빅토르 오르반 헝가리 총리를 비밀리에 만난 것으로 알려졌다.
오르반 총리는 헝가리의 민주주의를 몇 발짝 퇴보시켰다고 평가받는 인물이다.

% Biden, who would be 86 years old at the end of his next term, is working to assure a skeptical electorate that he’s still physically and mentally able to thrive in the world’s most important job.
한편 재선 임기가 끝나면 86세가 되는 바이든 대통령은 아직 미국 대통령이라는 중요한 자리에 설 만큼 자신이 건강하다고 유권자들을 안심시키는 데 여념이 없다.
% Voters in both parties are unhappy with his handling of immigration and inflation.
양당 지지자 모두 바이든 대통령의 외국인 이주자 및 인플레이션 억제 정책에는 불만을 표출했다.

% And he's dealing with additional dissension within his party’s progressive base, furious that he hasn’t done more to stop Israel’s war against Hamas in Gaza.
또한 바이든 대통령은 이스라엘--하마스 전쟁에 대한 미온적인 대처로 민주당 내 진보 세력의 뭇매를 맞고 있다.
% Activists and religious leaders in Washington encouraged Democrats to vote “uncommitted” to signal their outrage.
워싱턴 주의 운동가와 종교계는 민주당원들에게 기권표를 던져서 불만을 표출하자고 캠페인을 펼치기도 했다.

% In Seattle, 26-year-old voter Bella Rivera said they hoped their “uncommitted” vote would would serve as a wakeup call for the Democratic party.
시애틀에서 어린이집 선생님으로 근무하는 벨라 리베라(26) 씨는 자신의 ``기권'' 표가 민주당의 경각심을 일깨우는 계기가 되기를 바란다고 말했다.

% “If you really want our votes, if you want to win this election, you’re going to have to show a little bit more either support of Palestinian liberation — that’s something that’s very important to us — and ceasing funds to Israel,” said Rivera, a preschool teacher who uses they/them pronouns.
리베라 씨는 ``민주당이 진심으로 표를 원한다면, 정말로 선거에서 승리하고 싶다면 적어도 팔레스타인 해방을 위해 힘쓰거나 이스라엘 지원을 그만두어야 할 것''이라며 ``그게 유권자들에게 있어서 무척이나 중요한 문제''라고 설명했다.

% Almost 3,000 miles away in Georgia, retiree Donna Graham said she would have preferred another Republican nominee over Trump, but she said there's no way she'd ever vote for Biden in the general election.
은퇴 후 미 대륙 반대편인 조지아 주에서 거주 중인 돈나 그레이엄 씨는 트럼프 전 대통령 이외 다른 후보가 선출되었으면 했지만, 바이든 대통령에게는 절대로 표를 줄 생각이 없다고 단언했다.

% “He wasn’t my first choice, but he’s the next best thing,” Graham said of Trump. “It’s sad that it’s the same old matchup as four years ago."
그레이엄 씨는 ``트럼프는 최선의 선택은 아니지만, 그래도 차선책이다''라며 ``4년 전과 또 똑같은 선거가 반복된다니 정말 슬프다''고 토로했다.
\end{document}
