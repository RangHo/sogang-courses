\documentclass{translation}

\firstname{Juhun}
\lastname{Lee}
\instructor{Professor Hyejin Kim}
\course{Translation Practicum}
\date{13 March 2024}

% What's next for South Korean doctors who face license suspension because of walkouts
\title{면허 정지 위기에 놓인 의사들... 집단 파업, 그 다음은?}

\begin{document}
% SEOUL, South Korea (AP)
% South Korea’s government is pressing ahead with its vow to suspend the licenses of thousands of junior doctors who ignore its repeated demands to end their collective walkouts.
정부가 의대 증원 방침에 반발해 집단 사직서를 제출한 전공의 수천 명의 의사 면허를 정지하겠다는 확고한 의사를 밝혔다.
% Nearly 9,000 out of the country’s 13,000 medical interns and residents have been refusing to work for about two weeks to protest a government plan to increase South Korea’s medical school admission quota by about two thirds.
현 의대 정원에 2/3을 증원하겠다는 정부 방침에 맞서 13,000명의 전공의(인턴, 레지던트) 가운데 9,000여명이 병원을 떠난지 2주 가량이 지난 가운데,
% Here are some questions and answers about what’s next in the strike:
의사 파업 사태는 지금 어떻게 흘러가고 있을까.

% HOW DOES THE SUSPENSION WORK?
\section*{\( \diamondsuit \) 전공의 파업, 현재 상황은?} 
% After their walkouts caused hundreds of surgeries and other treatments to be canceled, the government ordered the junior doctors to return to work by Feb. 29 or face license suspensions and possible legal charges. Most of them missed the deadline.
전공의 파업으로 전국에서 수백 건의 진료 및 수술이 취소되자, 정부는 해당 의사들에게 2월 29일까지 진료를 재개하라는 업무개시명령을 송달했다.
미복귀 시 의사면허 정지를 포함한 법적 조치를 진행하겠다는 정부의 통보에도 상당수는 명령에 불응했다.

% On Monday, the government dispatched officials to about 50 hospitals to formally confirm the absence of striking doctors, before informing them of their license suspensions and giving them a chance to respond.
지난 26일, 정부는 대형 병원 50여 군데에 공무원을 파견해 전공의 파업 실태를 확인하도록 지시했다.
병원을 떠난 의사들에게 행정 처분에 대한 사전 통보를 함과 동시에 제재 전에 진료 복귀를 독려하기 위한 지침으로 보인다.

% Vice Health Minister Park Min-soo said the doctors face a minimum three-month suspension. Suspension records will leave them facing more than one year of delay in getting licenses for specialists and further barriers in landing jobs, Park said.
박민수 보건복지부 제2차관은 미복귀 전공의들에게 ``최소 3개월의 면허정지 처분이 불가피하다''며, ``이는 전문의 자격 획득에 1년 가량 제동을 걸 뿐만 아니라, 이후 취업에도 큰 걸림돌이 될 것''이라고 설명했다.
% Park suggested it would take weeks to complete procedures for suspending licenses. Once it’s done, some striking doctors will likely respond with legal action.
이어 박 차관은 ``행정 처분에는 적어도 몇 주는 필요하다''며, ``전공의들과의 법적 공방으로 번질 가능성도 인지하고 있다''고 덧붙였다.

% Hyeondeok Choi, partner at the law firm Daeryun that specializes in medical law, said it would be “impossible” for the government to suspend the licenses of all the 9,000 doctors. He said the government would likely target less than 100 of the leading strikers.
최형덕 법무법인 대륜 소속 의료전문변호사는 ``사직서를 제출한 전공의 9,000명 모두에게 면허정지 처분을 내리는 것은 사실상 불가능''하고, ``파업을 주도한 의사 대표 약 100명을 대상으로만 행정 처리를 강행할 것''으로 보인다고 의견을 밝혔다.

% The Korea Medical Association, which represents 140,000 doctors in South Korea, said it supports the junior doctors’ walkouts. Joo Sooho, a spokesperson at the KMA’s emergency committee, said Monday that senior doctors are considering economic support for the strikers if their licenses are suspended.
반면 대한의사협회(이하 의협)는 전공의 파업에 대한 전폭적인 지지 의사를 밝혔다.
주수호 의협 비상대책위원회 언론홍보위원장은 지난 26일 ``면허정지 처분을 받은 전공의에 대한 경제적 지원 방침을 교수들과 함께 고려 중''이라고 말했다.

% WHAT OTHER STEPS THE STRIKERS CAN FACE?
\section*{\( \diamondsuit \) 법적 제재에 직면한 전공의} 
% South Korea’s medical law says doctors who refuse the government’s back-to-work order can face up to three years in prison or a 30 million won ($22,480) fine, as well as up to one year of license suspensions. Those sent to prison or given even suspended prison sentences automatically lose their licenses.
현행 의료법상, 의사가 정부의 업무개시명령을 위탄한 경우 1년 이하의 면허정지 처분과 3년 이하의 징역, 또는 3000만원 이하 벌금에 처해질 수 있다.
% The Health Ministry can file complaints with police, who then investigate and hand the case to prosecutors for a possible indictment, according to Choi, the law firm partner.
최 변호사는 복지부가 경찰에 해당 의사를 고발하고, 이어 검찰에 기소까지도 가능하다고 분석했다.

% Joo said the Korea Medical Association will provide lawyers to the striking doctors if they are summoned by police or prosecutors.
이에 의협에서는 검\cdot경에 소환 조사를 받는 의사들에게 변호사 선임을 지원하겠다고 나섰다.
% South Korean police said they are investigating five senior members of the Korea Medical Association, after the Health Ministry filed complaints against them for allegedly inciting and abetting the junior doctors’ walkouts.
경찰은 전공의 집단사직을 교사한 혐의로 고발된 의협 지도부 5명에 대한 조사가 진행 중이라고 밝혔다.

% WHAT DO PEOPLE SAY?
\section*{\( \diamondsuit \) 차갑게 식은 여론 반응} 
% The doctors’ strikes have so far failed to generate public support, with a survey showing about 80% backing the government’s school enrollment plan.
한편 전공의 파업에 대한 여론 반응은 차갑다.
한 여론조사에 따르면, 국민 중 약 80\%가 정부의 의대 증원 계획에 찬성한다는 입장을 내비쳤다.

% The government says South Korea urgently needs more doctors to deal with a rapidly aging population. Many doctors say a too-steep increase in the number of students would eventually result in undermining medical service. Some critics say doctors, one of the highest-paid professions in South Korea, worry about losing their income.
정부는 빠르게 고령화 사회로 바뀌고 있는 현 상황에서 의사 증원이 급선무라는 입장을 고수하고 있다.
반면 의사 측은 급작스러운 의대 증원은 결국 의료 체계를 붕괴시킬 것이라며 맞섰다.
이미 연봉 최상위층에 위치한 의사들의 ``자기 밥그릇 챙기기''라는 비판도 잇달았다.

% Lee Yeonha, 40, said the striking doctors were “too selfish” and a three-month license suspension is too little.
서울에 거주하는 이연하(40)씨는 ``전공의들이 너무 이기적이다''며, ``면허정지 3개월 처분은 너무 가볍다''고 비판했다.
% “I wish the government would take more powerful legal action to get the doctors to fear that they may not be able to work as doctors in this country,” Lee said.
이씨는 ``의사들이 한국에서 더는 의사 생활을 못 하겠다고 느낄 정도로 강력한 법적 제재가 필요하다''며 정부의 대응을 촉구했다.

% Another Seoul resident, Sunny Shin, supports the arguments by doctors that the government must first resolve fundamental problems such as a lack of medical liability protection and a shortage of physicians in key yet low-paying specialties such as pediatrics and emergency departments.
반면, 다른 서울 시민 신서니씨는 ``현행 의료 체계에 대한 본질적인 문제 해결이 선행되어야 한다''면서, ``소아청소년과나 응급의료학과 등 필수 의료 인력에 대한 임금이나 책임 소재에 관한 문제는 아직도 남아있다''고 꼬집었다.
% “As long as the crucial sector doctors are likely to be embroiled in lawsuits and still not highly paid, I cannot blame them for protesting against the government labeling them as privileged people neglecting their duties as doctors,” Shin said.
신씨는 ``필수 의료 인력이 의료 소송과 저임금에 시달리는 현 상황에서 정부가 전공의들에게 씌운 `직무를 유기한 특권층'이라는 프레임에는 동의할 수 없다''고 덧붙였다.
\end{document}
